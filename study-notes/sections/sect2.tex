\section{Forward and Futures Contracts}
\label{sect:fwd-futures}
\subsection{Introduction}
\begin{enumerate}
\item A \defn{forward (contract)} is a contract between two parties (called
\defn{counterparties}) to buy or sell an asset (known as \defn{underlying
asset}, or simply \defn{underlying}) \faIcon{apple-alt} at a certain time in
the future (known as \defn{delivery date} or \defn{maturity date}) for a
specified price (known as \defn{forward price} or \defn{delivery price})
\faIcon{dollar-sign} (such that it costs nothing to enter the contract; see
\labelcref{it:fwd-cost-zero}).

\begin{remark}
\item The contract is negotiated, agreed, and signed \emph{today} (now). All
future transaction details are then \emph{fixed}.
\item We sometimes use the phrase ``forward \emph{on} \faIcon{apple-alt}'' to
describe the underlying asset is \faIcon{apple-alt}.
\end{remark}

\item A forward contract may be contrasted with a \emph{spot contract}:
\begin{itemize}
\item spot contract: two parties simply transact \faIcon{apple-alt} \emph{now}
at current market (spot) price (current ``fair'' price);
\begin{center}
\begin{tikzpicture}
\draw[-Latex] (0,0) -- (5,0) node[right] {Time};
\fill[] (0,0) circle [radius=0.05]
node[below=0.1cm, violet] (buy) {\faIcon{user-tie}}
node[above=0.1cm, orange] (sell) {\faIcon{user-tie}};
\node[] () at (0,-1) {\faIcon{scroll}\faIcon{check}};
\draw[-Latex] (buy.west) to[bend left] (sell.west);
\node[] () at (-1.1,0) {spot \faIcon{dollar-sign}};
\draw[-Latex] (sell.east) to[bend left] (buy.east);
\node[] () at (0.7,0) {\faIcon{apple-alt}};
\end{tikzpicture}
\end{center}
{\color{violet}\faIcon{user-tie}}: having a long position in \faIcon{apple-alt}\\
{\color{orange}\faIcon{user-tie}}: having a short position in \faIcon{apple-alt}\\
\item forward contract: two parties transact \faIcon{apple-alt} at a certain
future time point at a price that is ``fair'' deemed by now.
\begin{center}
\begin{tikzpicture}
\draw[-Latex] (0,0) -- (5,0) node[right] {Time};
\fill[] (0,0) circle [radius=0.05]
node[below=0.1cm, violet] (buy) {\faIcon{user-tie}}
node[above=0.1cm, orange] (sell) {\faIcon{user-tie}};
\draw[very thick, decorate,decoration={calligraphic brace, amplitude=5pt}] (buy.west) -- (sell.west)
node[midway, left=0.1cm] {negotiate \faIcon{scroll}};

\fill[] (3,0) circle [radius=0.05]
node[below=0.1cm, violet] (buyf) {\faIcon{user-tie}}
node[above=0.1cm, orange] (sellf) {\faIcon{user-tie}};
\draw[-Latex] (buyf.west) to[bend left] (sellf.west);
\node[] () at (1.9,0.2) {fwd.\ \faIcon{dollar-sign}};
\draw[-Latex] (sellf.east) to[bend left] (buyf.east);
\node[] () at (3.7,-0.2) {\faIcon{apple-alt}};
\node[] () at (3,-1) {\faIcon{scroll}\faIcon{check}};
\end{tikzpicture}
\end{center}
Conventionally, we regard ``owning a positive/negative amount of forward
\faIcon{scroll} (on \faIcon{apple-alt})'' as ``(having the obligation for) owning
the same amount, but of \faIcon{apple-alt}, \emph{at the delivery date}''.

By this convention, we know:
\begin{itemize}
\item {\color{violet}\faIcon{user-tie}} is having a long position in forward (or
long forward) (as {\color{violet}\faIcon{user-tie}} is owning a positive amount
of \faIcon{apple-alt} at the delivery date);
\item {\color{orange}\faIcon{user-tie}} is having a short position in forward
(or short forward) (as {\color{orange}\faIcon{user-tie}} is owning a negative
amount of \faIcon{apple-alt} at the delivery date).
\end{itemize}
Furthermore, closing out a position in forward (i.e., doing something such
that zero forward is owned) would also make the position ``at the delivery
date'' closed out (as zero \faIcon{apple-alt} would need to be
owned at that date).
\end{itemize}

\item \label{it:fwd-cost-zero}
An important feature of a forward is that there is no cost for the act of
``\emph{entering}'' into a forward contract (i.e., take a long/short position
in forward), ignoring transaction costs. Consequently, the \emph{value} of a
long/short position in forward is always zero \emph{at time 0}. (But the same
cannot be said for time point later than 0: See \cref{subsect:value-fwd}.)

\begin{remark}
\item This is just like the case for spot contract (i.e., contract for
buying/selling \faIcon{apple-alt} \emph{now}) --- Merely ``executing'' orders
itself would not cost anything, ignoring transaction costs.

\item It turns out that under the \emph{no-arbitrage principle} and
\emph{perfect market}, there is only one possible negotiated forward price (at
least for the assets covered in \cref{sect:fwd-futures-price}).
\end{remark}

\item A forward contract is an over-the-counter instrument (traded in OTC
market). A \defn{futures (contract)} has the same contract nature as a forward
contract, but it is an exchange-traded instrument (so all contract terms are
standardized).

\begin{note}
For a futures contract, the word ``forward'' is changed to ``futures'' in the
terminologies. For example:
\begin{itemize}
\item forward price \faIcon{arrow-right} futures price
\item long/short position in forward \faIcon{arrow-right} long/short position
in futures
\end{itemize}
\end{note}
\end{enumerate}
\subsection{Profit and Loss of a Position in Forward/Futures}
\begin{enumerate}
\item The \defn{profit and loss} (\defn{P/L} or \defn{P\&L}) of a (long/short)
position at time \(t\) is the payoff of the position at time \(t\) less the
\emph{future value} of previous cash \emph{outflows} (before time \(t\)) at
time \(t\) (at the \emph{risk-free interest rate}).
\begin{center}
\begin{tikzpicture}
\draw[-Latex] (0,0) -- (5,0) node[right] {Time};
\fill[] (3,0) circle [radius=0.05]
node[above right] {\(t\)};
\draw[->, red, thick, opacity=0.2] (0,-0.2) -- (0,0.2)
node[pos=1.5]{\faIcon{dollar-sign}};
\draw[->, red, thick] (2.95,-0.2) -- (2.95,0.2)
node[pos=1.5]{\faIcon{dollar-sign}\faIcon{dollar-sign}};
\draw[->, brown, thick] (3.05,0.2) -- (3.05,-0.2)
node[pos=1.5]{\faIcon{dollar-sign}\faIcon{dollar-sign}\faIcon{dollar-sign}};
\node[] () at (2.5,-1) {\faIcon{user-tie}};
\draw[-Latex, violet] (0,0.7) to[bend left] (3,0.7);
\end{tikzpicture}
\end{center}

\begin{note}
P/L at time \(t\) is the net cash flow at time \(t\), after ``accumulating''
(without risk) cash flows\footnote{For positive cash outflow (inflow) at a
previous time, first borrow (lend) that amount risk-free (to ``cancel out''
that cash flow), and then repay (collect proceeds from) the loan at time \(t\)
\faIcon{arrow-right} yielding a cash outflow (inflow) with amount equal to the
future value at time \(t\). See \cref{sect:fwd-futures-price} for more examples
like this.} from previous time points if needed. This indicates how much
\emph{profit} can be earned from ``just'' that position (without ``adding''
extra risk).
\end{note}

\item \label{it:fwd-futures-notations}
Some notations:
\begin{itemize}
\item \(S_t\): (spot) price of one unit of underlying asset
\faIcon{apple-alt} at time \(t\);
\item time \(T\) (positive): delivery date;
\item \(K\): delivery price.
\end{itemize}


\item \label{it:pl-long-fwd}
The P/L (at time \(T\)) of a long position in forward/futures on one unit of
\faIcon{apple-alt} is \(S_T-K\) (which equals its payoff at time \(T\) as there
are no cash flows before time \(T\)).

\begin{pf}
To close out all positions at time \(T\), perform:
\begin{center}
\begin{tabular}{lcr}
\toprule
Transaction&Position change&Cash flow\\
\midrule
buy 1 \faIcon{apple-alt} at the delivery price&
1 \faIcon{scroll} \& 0 \faIcon{apple-alt} \faIcon{arrow-right} 0 \faIcon{scroll} \& 1 \faIcon{apple-alt}
&\(-K\)\\
sell 1 \faIcon{apple-alt} at the spot price
&1 \faIcon{apple-alt} \faIcon{arrow-right} 0 \faIcon{apple-alt}
&\(+S_T\)\\
&&Total: \(S_T-K\)\\
\bottomrule
\end{tabular}
\end{center}
\end{pf}
\begin{center}
\begin{tikzpicture}
\begin{axis}[domain=0:5, ymin=-3, ymax=3, axis y line=left, axis x line=middle,
xtick={2}, xticklabels={\(K\)}, ytick=\empty, ylabel=P/L,
ylabel style={at={(axis description cs:0,1)}, anchor=south, rotate=-90}, 
xlabel=\(S_T\),
xlabel style={at={(axis description cs:1,0.5)}, anchor=west}
]
\addplot[blue]{x-2};
\draw[very thick, decorate,decoration={mirror, calligraphic brace, amplitude=5pt, raise=5pt}] (2.05,0) -- (5,0)
node[midway, below=0.3cm] {profitable};
\node[blue] () at (3,2) {slope = 1};
\end{axis}
\end{tikzpicture}
\end{center}
When \(S_T\uparrow\), the long position makes money. So, speculators can \emph{long
forward/futures} to bet \faIcon{coins} the price of \faIcon{apple-alt} rises
\faIcon{chart-line} in the future.

\item \label{it:pl-short-fwd}
The P/L of a short position in forward/futures on one unit of
\faIcon{apple-alt} is \(K-S_T\) (which equals its payoff at time \(T\)).

\begin{pf}
An one-line proof is that it follows from the property that payoff (value) of
the short forward/futures at time \(T\) is simply the negative of \(S_T-K\)
(payoff of the long forward/futures).
Alternatively, consider the following.

To close out all positions at time \(T\), perform:
\begin{center}
\begin{tabular}{lcr}
\toprule
Transaction&Position change&Cash flow\\
\midrule
(short) sell 1 \faIcon{apple-alt} at the delivery price
&\(-1\) \faIcon{scroll} \& 0 \faIcon{apple-alt} \faIcon{arrow-right} 0 \faIcon{scroll} \& \(-1\) \faIcon{apple-alt}
&\(+K\)\\
buy 1 \faIcon{apple-alt} at the spot price
&\(-1\) \faIcon{apple-alt} \faIcon{arrow-right} 0 \faIcon{apple-alt}
&\(-S_T\)\\
&&Total: \(K-S_T\)\\
\bottomrule
\end{tabular}
\end{center}
\end{pf}
\begin{center}
\begin{tikzpicture}
\begin{axis}[domain=0:5, ymin=-3, ymax=3, axis y line=left, axis x line=middle,
xtick={2}, xticklabels={\(K\)}, ytick=\empty, ylabel=P/L,
ylabel style={at={(axis description cs:0,1)}, anchor=south, rotate=-90}, 
xlabel=\(S_T\),
xlabel style={at={(axis description cs:1,0.5)}, anchor=west}
]
\addplot[blue]{2-x};
\draw[very thick, decorate,decoration={mirror, calligraphic brace, amplitude=5pt, raise=5pt}] (0,0) -- (1.95,0)
node[midway, below=0.3cm] {profitable};
\node[blue] () at (2,1.5) {slope = \(-1\)};
\end{axis}
\end{tikzpicture}
\end{center}
\begin{note}
In general, since the payoff (and P/L also indeed) of the short position is
just the negative of that for the long position, the short position payoff (or
P/L) graph is just the mirror image of the long position one, across the
\(S_T\)-axis.
\end{note}

When \(S_T\downarrow\), the short position makes money. So, speculators can
\emph{short forward/futures} to bet \faIcon{coins} the price of
\faIcon{apple-alt} drops \faIcon{chart-area} in the future.
\end{enumerate}

\subsection{Stock Index Futures}
\begin{enumerate}
\item A \defn{stock index} tracks changes in the value of a hypothetical
portfolio \faIcon{chart-pie} of stocks. So it can be regarded as a weighted
average of prices of different stocks. Example: S\&P 500.

\item A \defn{stock index futures} is a futures on stock index. Since stock
index cannot be ``delivered'' physically, we have \defn{cash settlement} for
those futures, i.e., investors are required to \emph{close out} their positions
in those futures and receive/pay cash \faIcon{dollar-sign} at or before
maturity, and there is no (physical) ``delivery'' at maturity).

\begin{remark}
\item To close out a long (short) position \emph{at} delivery date, we simply
fulfill the obligation suggested in the futures \faIcon{scroll}, i.e., buy
(sell) \faIcon{apple-alt} at the specified price \faIcon{dollar-sign}.

\item To close out a long (short) position in the futures \faIcon{scroll}
\emph{before} delivery date, we can take a short (long) position in the
\emph{same} futures \faIcon{scroll}. Since it is negotiated at time 0, its
value at time \(0<t<T\) may \emph{not} be zero anymore. (See
\cref{subsect:value-fwd} for more details.)
\end{remark}

\end{enumerate}

\subsection{Short and Long Hedge}
\begin{enumerate}
\item A \defn{hedge} \faIcon{shield-alt} is a trade designed to reduce risk \faIcon{fire-alt}. A
\defn{perfect hedge} is a hedge that \emph{completely} eliminates the risk
\faIcon{fire-alt}.
\item A \defn{short hedge} (\defn{long hedge}) is a hedge involving a short
(long) position in forward/futures.
\item Situations where \emph{short hedges} are useful:
\begin{itemize}
\item the hedger already owns an asset \faIcon{apple-alt} and expects to sell
\faIcon{apple-alt} at a certain future time point \faIcon{clock};
\item the hedger does not own \faIcon{apple-alt} now, but will own
\faIcon{apple-alt} later, and then sell \faIcon{apple} at a certain future time
point \faIcon{clock}.
\end{itemize}
In either situation, a \emph{short hedge} can let the hedger \emph{lock in}
\faIcon{lock} the selling price \faIcon{dollar-sign} (namely the price specified
in forward/futures \faIcon{scroll}) \emph{now}, completely eliminating the
uncertainty of future selling price \faIcon{arrow-right} \emph{perfect hedge}.

\item Situation where \emph{long hedges} are useful:
\begin{itemize}
\item the hedger has to purchase \faIcon{apple-alt} at a certain future time
point \faIcon{clock}.
\end{itemize}
Likewise, a \emph{long hedge} can let the hedger \emph{lock in}
\faIcon{lock} the purchasing price \faIcon{dollar-sign} \emph{now}, completely
eliminating the uncertainty of future purchasing price \faIcon{arrow-right}
\emph{perfect hedge}.
\end{enumerate}
