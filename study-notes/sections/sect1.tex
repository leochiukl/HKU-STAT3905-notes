\section{Introduction to Derivatives}
\label{sect:intro-deriv}
\subsection{Definition and Uses of Derivatives}
\begin{enumerate}
\item A (financial) \defn{derivative} is a contract \faIcon{scroll} (between
two parties) whose value (or payoff)\footnote{We define value/payoff in
\cref{subsect:long-short-pos}.} depends on (or ``is derived from'') the value(s)
of other more basic underlying variable(s) (e.g., total rainfall
\faIcon{cloud-rain} over a certain period in the future).

\item Examples of derivatives: futures, forwards, options, swaps, insurance.
(The first four are discussed in later sections. For insurance, see e.g.,
STAT3901.)

\item Uses of derivatives:
\begin{itemize}
\item risk management \faIcon{shield-alt}: reduce the overall level of risk
\faIcon{fire-alt} (example: insurance);

\item speculation \faIcon{dice} (risky!
{\color{red}\faIcon{exclamation-triangle}}): make bets \faIcon{coins} on
various market quantities \faIcon{chart-line};

\item reducing transaction costs \faIcon{file-invoice-dollar}
\faIcon{arrow-alt-circle-down}: \emph{replicate} \faIcon[regular]{copy} a
series of transactions
\faIcon[regular]{handshake} \faIcon[regular]{handshake} \faIcon[regular]{handshake}
by a \emph{single} transaction \faIcon[regular]{handshake} of a derivative
\faIcon{scroll}, so that less transactions are made, thereby reducing
transaction costs;

\item regulatory arbitrage \faIcon{crosshairs}: circumvent regulatory
restrictions \faIcon{ban} [e.g., banks \faIcon{landmark} securitized mortgages
\faIcon{house-user} and bought back the securitized products created, since the
capital required to be kept for those securitized products used to be much less
than that for the mortgages themselves \parencite[Section~8.3]{hull2022options}
(such products led to \emph{global financial crisis} \faIcon{bomb} in 2008!)].
\end{itemize}
\end{enumerate}

\subsection{Exchange-Traded and Over-The-Counter Markets}
\begin{enumerate}
\item An \defn{exchange-traded market} \faIcon{network-wired} is a market
where standardized financial instruments \faIcon{scroll} are traded
\faIcon[regular]{handshake}.

\item An \defn{over-the-counter market} (OTC market) \faIcon{people-arrows} is
a market where participants \faIcon{user-tie} trade (possibly tailor-made
\faIcon{pencil-ruler}) financial instruments \faIcon{scroll} directly with each
other, without a central exchange.

\item For the exchange-traded market \faIcon{network-wired}, usually there are
\emph{margins requirement} \faIcon{donate} and \emph{marking to market}
\faIcon{redo}, to eliminate \defn{credit risk} \faIcon{running} (i.e., the risk
that the contract \faIcon{scroll} will not be honored).

\item The actual implementations of margins requirement \faIcon{donate} and
marking to market may vary for different exchange-traded markets. But the
general idea is as follows:
\begin{itemize}
\item margins requirement: to enter into a trade \faIcon[regular]{handshake},
both parties need to \emph{deposit} \faIcon{donate} a certain amount into a
\emph{margin account};
\item marking to market: as the market price \faIcon{dollar-sign} of the traded
instrument varies \faIcon{chart-line}, the balances needed in the margin
accounts also update accordingly at a certain frequency (e.g., daily).
\end{itemize}
\begin{note}
Often there is a prespecified ``minimum balance'' for the margin account
(called \defn{maintenance margin}). If the account balance is lower than this
value, further deposit \faIcon{donate} is required.
\end{note}

\item With suitable margins requirement and marking to market, even if a party
\faIcon{user} does not honor the contract (``runs away'') \faIcon{running},
another party can still receive a compensation (the balance in \faIcon{user}'s
margin account) that covers the loss, thereby eliminating credit risk.

\item On the other hand, there are usually not margins requirement and marking
to market for OTC market, so credit risk exists.
\end{enumerate}

\subsection{Types of Traders}
\begin{enumerate}
\item There are three main types of traders in a derivatives market:
\begin{itemize}
\item hedgers: use derivatives to reduce risk \faIcon{shield-alt} that they face
from potential future movements \faIcon{chart-line} in market quantities;
\item speculators: use derivatives to bet \faIcon{coins} on future directions
of market quantities \faIcon{chart-line} --- they try to make money
\faIcon{dollar-sign} by taking risk \faIcon{fire-alt};
\item arbitrageurs: take offsetting ``positions''\footnote{This means
aggregating all those positions gives \emph{zero} unit of instrument. In other
words, the positions are completely ``closed out''. See
\cref{subsect:long-short-pos} for more details.} in two or more instruments to
lock in \faIcon{lock} a sure (risk-free) profit.
\end{itemize}

\item A \defn{bullish} (\defn{bearish}) trader expects that a certain market
quantity (often price of an asset \faIcon{apple-alt}) will \emph{rise}
{\color{ForestGreen}\faIcon{chart-line}} (\emph{drop}
{\color{red}\faIcon{chart-area}}) in the future. With such expectation, the
trader may then use appropriate strategy involving derivatives to make bet
\faIcon{coins}. (See \cref{subsect:bull-bear-speculate}.)

\begin{mnemonic}
A bull uses its horns in an \emph{upward} \faIcon{level-up-alt} motion to attack and a bear uses its
claws in a \emph{downward} \faIcon{level-down-alt} motion to attack.
\end{mnemonic}


\item The act of (possibly) making a risk-free profit without needing any (net) cash
outflow (having a ``free lunch'') is known as \defn{arbitrage}. Since taking
offsetting positions would not lead to any (net) cash outflow, an
\emph{arbitrage opportunity} arises if such act can lead to a risk-free profit.

\item \label{it:no-arbitrage-principle}
In modern financial economics, a fundamental assumption is the
\defn{no-arbitrage principle}, i.e., arbitrage opportunities \emph{do not
exist}. This assumption may be regarded as ``reasonable'' if the arbitrageurs
``act immediately'' to capture the arbitrage opportunity once it arises, making the
opportunity vanish almost ``instantaneously'' (through market demand and supply).

In this notes, we shall assume that this principle holds true unless stated
otherwise.
\end{enumerate}

\subsection{Buying and Selling Financial Instruments}
\begin{enumerate}
\item To buy/sell financial instruments, typically we do so through a
\emph{broker} (e.g., banks \faIcon{landmark}), in an exchange-traded market
\faIcon{network-wired}.  We also need to pay transaction costs
\faIcon{file-invoice-dollar}, e.g., brokerage commission, taxes, etc.

\item In an exchange-traded market, participants place \defn{orders} (i.e.,
instructions to buy/sell a certain number of financial instruments). There are
two main kinds of orders:
\begin{itemize}
\item \defn{market order}: the ordered transaction is to be executed immediately at
current ``market price'';
\item \defn{limit order}: the instruments are to be bought (sold) at no more
(no less resp.) than a specified price (called \defn{limit price}).
\end{itemize}

\item Limit orders for a financial instrument may be visualized as follows (the
numbers indicate the limit prices specified by the participants
\faIcon{user}):\footnote{We shall assume that there are at least one buy order
and at least one sell order. (The market is not too ``illiquid''.)}
\begin{center}
\begin{tikzpicture}
\draw[-Latex] (0,0) -- (0,8)
node[pos=1.05] {Price};
\node[] () at (-2.5,8) {Buy orders};
\node[] () at (2.5,8) {Sell orders};
\node[] () at (-3,1) {\faIcon{user} \faIcon{user} \faIcon{user}};
\node[] () at (-1,1) {9.25};
\node[] () at (-3,2) {\faIcon{user}};
\node[] () at (-1,2) {9.5};
\node[] () at (-3,4) {\faIcon{user} \faIcon{user} \faIcon{user}};
\node[violet] () at (-1,4) {10};
\node[violet] () at (-5,4) {bid price};
\node[] () at (-1,3) {9.75};
\node[] () at (3,5) {\faIcon{user} \faIcon{user}};
\node[orange] () at (1,5) {10.25};
\node[orange] () at (5,5) {ask price};
\node[] () at (3,6) {\faIcon{user} \faIcon{user} \faIcon{user}};
\node[] () at (1,6) {10.5};
\node[] () at (3,7) {\faIcon{user} \faIcon{user} \faIcon{user} \faIcon{user}};
\node[] () at (1,7) {10.75};
\draw[<->, very thick, blue] (0,4) -- (0,5)
node[midway, font=\small] {bid-ask spread};
\end{tikzpicture}
\end{center}

\item In the visualization, the ``top'' (``bottom'') price at LHS (RHS) is
known as \defn{bid price} (\defn{ask price} or \defn{offer price}), which is
the price where a new participant \faIcon{user-tie} can \emph{sell} (\emph{buy}) the
instrument \emph{immediately} (the execution price of a sell (buy) market
order).

\begin{note}
Bid price is called ``bid'' as it sources \emph{from} the ``bidding'' (buying)
side, and ask/offer price is called ``ask/offer'' as it sources \emph{from} the
``asking/offering'' (selling) side.
\end{note}

\begin{warning}
\emph{Buy} market order executes at \emph{ask} price, and \emph{sell} market
order executes at \emph{bid} price. Do not mix up them!
\end{warning}

\item The difference between bid price and ask price is known as \defn{bid-ask spread}.

\item At any moment of time, ask price is greater than bid price. To see this,
consider what would happen if ask price was less than or equal to bid
price:\footnote{Here we suppose each limit order is to buy/sell one unit of the
instrument.}
\begin{center}
\begin{tikzpicture}
\draw[-Latex] (0,0) -- (0,8)
node[pos=1.05] {Price};
\node[] () at (-2.5,8) {Buy orders};
\node[] () at (2.5,8) {Sell orders};
\node[] () at (-3,1) {\faIcon{user} \faIcon{user} \faIcon{user}};
\node[] () at (-1,1) {9.25};
\node[] () at (-3,2) {\faIcon{user}};
\node[] () at (-1,2) {9.5};
\node[] () at (-3,4) {\faIcon{user} \faIcon{user} \faIcon{user}};
\node[] () at (-1,4) {10};
\node[] () at (-1,3) {9.75};
\node[] () at (-3,5) {\faIcon{user} \faIcon{user-slash} \faIcon{user-slash}};
\node[violet] () at (-1,5) {10.25};
\node[violet] () at (-5,5) {``bid price''};
\node[] () at (3,5) {\faIcon{user-slash} \faIcon{user-slash}};
\node[orange] () at (1,5) {10.25};
\node[orange] () at (5,5) {``ask price''};
\node[] () at (3,6) {\faIcon{user} \faIcon{user} \faIcon{user}};
\node[] () at (1,6) {10.5};
\node[] () at (3,7) {\faIcon{user} \faIcon{user} \faIcon{user} \faIcon{user}};
\node[] () at (1,7) {10.75};

\draw[brown] (-3,4.7) .. controls (-1,4) and (1,4) .. (3,4.7)
node[midway, above=0.1cm] {2 transactions};
\end{tikzpicture}
\end{center}
After executing the orders, the situation becomes:
\begin{center}
\begin{tikzpicture}
\draw[-Latex] (0,0) -- (0,8)
node[pos=1.05] {Price};
\node[] () at (-2.5,8) {Buy orders};
\node[] () at (2.5,8) {Sell orders};
\node[] () at (-3,1) {\faIcon{user} \faIcon{user} \faIcon{user}};
\node[] () at (-1,1) {9.25};
\node[] () at (-3,2) {\faIcon{user}};
\node[] () at (-1,2) {9.5};
\node[] () at (-3,4) {\faIcon{user} \faIcon{user} \faIcon{user}};
\node[] () at (-1,4) {10};
\node[] () at (-1,3) {9.75};
\node[] () at (-3,5) {\faIcon{user}};
\node[violet] () at (-1,5) {10.25};
\node[violet] () at (-5,5) {bid price};
\node[] () at (3,6) {\faIcon{user} \faIcon{user} \faIcon{user}};
\node[orange] () at (1,6) {10.5};
\node[orange] () at (5,6) {ask price};
\node[] () at (3,7) {\faIcon{user} \faIcon{user} \faIcon{user} \faIcon{user}};
\node[] () at (1,7) {10.75};
\draw[<->, very thick, blue] (0,5) -- (0,6)
node[midway, font=\small] {bid-ask spread};
\end{tikzpicture}
\end{center}
So, now the ask price is greater than bid price again. We may assume the
executions of such orders happen ``instantaneously'', and then it is not hard
to see that the only possible observation is ask price exceeding bid price.
\end{enumerate}

\subsection{Short Selling}
\begin{enumerate}
\item \defn{Short selling} means selling a financial instrument \faIcon{scroll}
that is not owned.
\item The procedure for short selling is as follows:
\begin{enumerate}
\item Borrow a number of instruments \faIcon{scroll} from a third party (e.g.
broker) \faIcon{landmark}.

\item Sell those instruments \faIcon{scroll} to the market \faIcon{network-wired}
for cash \faIcon{dollar-sign} (creating a \emph{short position}).

\item Use the cash \faIcon{dollar-sign} to buy that number of instruments
\faIcon{scroll} some time later \faIcon{clock} and return them to the lender
\faIcon{landmark} (\emph{closing out} or \emph{covering} the short position).
\end{enumerate}
\begin{note}
The lender \faIcon{landmark} usually requires the short-seller to deposit a
certain amount of money as a \emph{collateral}, to protect against the
possibility that the short-seller ``runs away'' \faIcon{running} and fails to
return the instruments \faIcon{scroll} when its price surges in the future.
\end{note}

\item Short selling may not be possible for certain instruments.

\item Speculators can short sell an instrument \faIcon{scroll} to
bet \faIcon{coins} its price to drop {\color{red}\faIcon{chart-area}} in the
future. They make money if the price indeed goes down, as they first ``sell
high'', then ``buy low''. But if the price goes up, they would first ``sell
low'' then ``buy high'', resulting in a loss.

\item An investor with a short position is required to pay to the lender
\faIcon{landmark} any income (e.g., dividends \faIcon{money-bill-wave}) that
would normally be received on the ``shorted'' instruments, since the \emph{owner}
(the lender), not the short-seller, is entitled to those incomes.
\end{enumerate}

\subsection{Long and Short Positions}
\label{subsect:long-short-pos}
\begin{enumerate}
\item \label{it:long-short-def}
Having a \defn{long position} (\defn{short position}) in a financial
instrument \faIcon{scroll} means owning a \emph{positive} (``\emph{negative}'')
amount of \faIcon{scroll}.

\begin{remark}
\item Owning a ``negative'' amount of \faIcon{scroll} actually means \emph{owing}
that amount (in absolute value) of \faIcon{scroll}.
\item For brevity, we may simply use the word ``long (short)'' to mean ``long
(short) position in''. Example: ``long \faIcon{scroll}'' means ``long position
in \faIcon{scroll}''.
\item Unless otherwise specified, a long (short) position in \faIcon{scroll}
means owning (owing) \emph{one unit} of \faIcon{scroll}.
\item An alternative expression is ``\defn{long} (\defn{short}) a number of
instrument \faIcon{scroll}'' (treating ``long''/``short'' as a verb). The
number specified is the increase (decrease) in amount of \faIcon{scroll} owned.
\end{remark}

\item Example: in the short selling, the short-seller \faIcon{user} owes an amount of
\faIcon{scroll} to the lender \faIcon{landmark} (so \faIcon{user} owns a
negative amount of \faIcon{scroll}). Hence, \faIcon{user} is having a
\emph{short} position.  (This also explains why it is called ``short''
selling.)

\item \defn{Closing out} or \defn{covering} a (long or short) position in a
financial instrument \faIcon{scroll} means doing something such that
\emph{zero} \faIcon{scroll} is owned (``clear'' the amount of \faIcon{scroll}
we own).

\item Example: if \faIcon{user} has a long position in \faIcon{scroll}, then
\faIcon{user} can short 1 \faIcon{scroll} to close out his long position.

\item The \defn{value} or \defn{payoff} of a position in a financial instrument
\faIcon{scroll} at a certain time is the amount of money \faIcon{dollar-sign}
received \emph{if} all positions (including those ``extra'' positions in other
instruments created from closing out the position in \faIcon{scroll}) are
closed out at that time.\footnote{Implicitly we assume that this amount is
\emph{unique}, and this is justified by the no-arbitrage principle (and also
the \emph{perfect market} assumption; see \labelcref{it:perfect-mkt}):
Different ways of closing out those positions must yield the same amount of
cash flow.  Otherwise, we can perform the way yielding higher cash flow and
perform the \emph{reverse} (``opposite'') of the way yielding lower cash flow
to ``capture'' the difference risk-free (arbitrage). See \labelcref{it:loop}
for more explanation on ``reverse''.}

\begin{remark}
\item Receiving a negative amount of \faIcon{dollar-sign} means \emph{paying}
that amount (in absolute value) of \faIcon{dollar-sign}.

\item ``Value of an amount of \faIcon{scroll}'' means ``value of \emph{owning}
that amount of \faIcon{scroll}''. It also has the meaning of ``(spot)
\emph{price} of that amount of \faIcon{scroll}'' since selling that amount
(yielding the (spot) \emph{price}) closes out the position of ``owning that
amount of \faIcon{scroll}''.

\item Unless stated otherwise, the time unit is always years in this notes.
\end{remark}
\item \label{it:buy-sell-any-num}
Unless stated otherwise, we shall assume one can always freely buy or sell any
number of financial instrument available in the market, at the (single)
prevailing \defn{spot price} (price per unit of the instrument at which
transaction can be done \emph{immediately}). (So the bid-ask spread is
ignored.) This may be ``reasonable'' if the market is ``very liquid''.

\begin{note}
We shall also assume the instruments we state in this notes (e.g. loan, stock
etc.) are all available in the market, unless stated otherwise.
\end{note}

\item \label{it:value-linear}
Under this assumption, we have the following ``linearity'' property for
value:
\[
\text{value of owning \(k\) \faIcon{scroll}} = k\times\text{value of owning 1 \faIcon{scroll}}.
\]
\begin{note}
We can also phrase it in the following way, which may be more intuitive:
\[
\text{price of \(k\) \faIcon{scroll}} = k\times\text{price of 1 \faIcon{scroll}}.
\]
\end{note}

\begin{pf}
Firstly, the value of owning 1 \faIcon{scroll} is the amount of cash
\faIcon{dollar-sign} received from selling 1 \faIcon{scroll} (a way of closing
the long position), i.e., spot price of \faIcon{scroll} (denoted by \(S\)
here). Now, since selling \(k\) \faIcon{scroll} results in a cash flow of
\(kS\) (they can all be bought/sold at the spot price \(S\) by assumption), the
value of owning \(k\) \faIcon{scroll} is \(kS\) also, by definition.

\begin{note}
When \(k\) is negative, ``selling \(k\) \faIcon{scroll}'' is supposed to
mean ``buying \(-k\) \faIcon{scroll}''. Also, positive (negative) cash flow
represents cash inflow (outflow).
\end{note}
\end{pf}

As a special case, the value of ``short \faIcon{scroll}'' is the negative of
the value of ``long \faIcon{scroll}''.
\end{enumerate}
