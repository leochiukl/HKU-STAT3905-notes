\section{Options}
\label{sect:options}
\begin{enumerate}
\item \emph{Options} are contracts that are somewhat similar to
forward/futures. They also allow us to buy/sell an underlying asset
\faIcon{apple-alt} at a specific price. The following are the main
distinctions:
\begin{itemize}
\item right vs.\ obligation: An option gives us \emph{right} to buy/sell
\faIcon{apple-alt} (we have an ``option'' to buy/sell \faIcon{apple-alt}
\emph{or not}), but a forward/future only gives us \emph{obligation} to buy/sell
\faIcon{apple-alt} (we \emph{must} buy/sell \faIcon{apple-alt} at delivery date
no matter what).
\item buy/sell timing: For some options, it is possible to buy/sell
\faIcon{apple-alt} even \emph{before} maturity (time \(T\)).
\end{itemize}
\end{enumerate}
\subsection{Call and Put Options}
\begin{enumerate}
\item There are two basic types of options: call and put options.

\item A \defn{call option}/\defn{put option} (or simply \defn{call}/\defn{put}
(resp.)) \faIcon{scroll} gives its holder the \emph{right} (but not the
obligation) to buy/sell (resp.) an underlying asset \faIcon{apple-alt} at a
specific price (known as \defn{exercise price} or \defn{strike price}), by a
certain date (known as \defn{expiration date} or \defn{maturity date}).

\begin{note}
The act of \emph{using} the right given by the option is known as
\defn{exercising the option}.
\end{note}

\begin{mnemonic}
For the call option, we have the option to ``call'' \faIcon{apple-alt} from
someone, and then own it. For the put option, we have the option to ``put'' our
\faIcon{apple-alt} to someone's hand \faIcon{hand-holding}.
\end{mnemonic}

\item For each of call and put options, it can be \emph{American} or
\emph{European}. Their difference is on the buy/sell timing:
\begin{itemize}
\item The holder of an \defn{American} option can exercise at any time on or before the expiration date.
\item The holder of an \defn{European} option can only exercise at the expiration date (``like'' forward/futures).
\end{itemize}
\begin{mnemonic}
\underline{A}merican: \underline{A}nytime; \underline{E}uropean: \underline{E}xpiration.
\end{mnemonic}

\begin{note}
Unless otherwise specified, options here are \emph{European}. (European options
turn out to be more mathematically tractable.)
\end{note}

\item Like forward/futures, we also have the long/short terminology for
options. Here the meaning follows the definition in
\labelcref{it:long-short-def}.
\begin{itemize}
\item holding/owning a positive amount of call (put) options \faIcon{scroll}
\faIcon{arrow-right} \emph{long} position in call (put) option, or in short,
long call (put)
\item holding/owning a negative amount of call (put) options
\faIcon{scroll}\footnote{That is, selling/\defn{writing} that amount (in
absolute value) of call (put) options \faIcon{scroll}.} \faIcon{arrow-right}
\emph{short} position in call (put) option, or in short, short call (put)
\end{itemize}

\item Due to the ``similarity'' to forward/futures, we shall use the notations
from \labelcref{it:fwd-futures-notations} for strike price and expiration date:
\begin{itemize}
\item \defn{\(K\)}: strike price (``like'' delivery price for forward/futures);
\item time \defn{\(T\)} (positive): expiration date (``like'' delivery date for
forward/futures).
\end{itemize}

\item In this notes, we assume that the option holder is always ``rational''
(profit-maximizing).
\end{enumerate}
\subsection{Payoff of Call and Put Options}
\begin{enumerate}
\item \label{it:euro-option-rational}
In the context of European options, rationality implies that a holder
would exercise the option (at time \(T\)) iff a \emph{positive} profit can be
earned from exercising at that time (and then closing out all positions).

\begin{note}
To avoid the ``indifferent'' situation, here we assume the holder would
\emph{not} exercise the option if there is zero profit to be earned.
\end{note}

\item Further elaboration on \labelcref{it:euro-option-rational}: At time \(T\)
there are the following two choices that are ``comparable'' (in the sense that
both result in no position):
\begin{enumerate}
\item do not exercise the option;
\item exercise the option and then close out all positions.
\end{enumerate}
Rationality suggests that between these two ``comparable'' choices, the option
holder would choose the one that results in a higher profit (more positive cash
flow). Since the former choice results in \emph{zero} cash flow, this means the
holder would choose the latter one iff it results in a \emph{positive} (\(>0\))
profit (cash flow).

\item \label{it:call-put-ex-criterion}
For a call (put) option \faIcon{scroll} on 1 \faIcon{apple-alt}, this means the
option would be exercised iff \(S_T>K\) (\(K>S_T\)).

\begin{pf}
Note that at time \(T\), exercising the call (put) and then closing out all
positions --- i.e., selling (buying) 1 \faIcon{apple-alt} at the spot price in
this context --- results in a total cash flow of \(S_T-K\) (\(K-S_T\)).
\end{pf}
\item \label{it:lc-lp-payoff}
Consequently, the payoff of the \emph{long} call (put) at time \(T\) is
\((S_T-K)_+\) (\((K-S_T)_+\) resp.), where \(x_{+}\) is the \defn{positive
part} of \(x\), i.e., \(\max\{x,0\}\) (or \(x\text{\,\defn{\(\vee\)}\,}0\)).

\begin{pf}
This immediately follows from the proof for
\labelcref{it:call-put-ex-criterion}: The act of exercising and then closing
out all positions afterward (or not exercising and letting the option expire)
is what we need to close out all positions at time \(T\), and the cash flow
obtained from such act is \((S_T-K)_+\) (\((K-S_T)_{+}\)) for call (put).
\end{pf}
\begin{center}
\begin{tikzpicture}[declare function={func(\x) = (\x <= 2) * 0 + (\x > 2) * (x-2);}]
\begin{axis}[domain=0:5, ymin=-1, ymax=3.5, xmax=5.5, axis y line=left, axis x line=middle,
xtick={2}, xticklabels={\(K\)}, ytick=\empty, 
ylabel=Payoff,
ylabel style={at={(axis description cs:0,1)}, anchor=south, rotate=-90},
xlabel=\(S_T\),
xlabel style={anchor=west}, samples=100, title=Long Call
]
\addplot[blue, thick]{func(x)};
\node[blue] () at (4,1) {slope = 1};
\end{axis}
\end{tikzpicture}
\begin{tikzpicture}[declare function={func(\x) = (\x <= 2) * (2-x) + (\x > 2) * 0;}]
\begin{axis}[domain=0:5, ymin=-1, ymax=3.5, xmax=5.5, axis y line=left, axis x line=middle,
xtick={2}, xticklabels={\(K\)}, ytick=\empty, ylabel=Payoff,
ylabel style={at={(axis description cs:0,1)}, anchor=south, rotate=-90},
xlabel=\(S_T\),
xlabel style={anchor=west}, samples=100, title=Long Put
]
\addplot[blue, thick]{func(x)};
\node[blue] () at (2,1) {slope = \(-1\)};
\end{axis}
\end{tikzpicture}
\end{center}
\item \label{it:sc-sp-payoff}
Then, the payoff of the \emph{short} call (put) at time \(T\) is
\(-(S_T-K)_{+}\) (\(-(K-S_T)_{+}\)) (negative of the payoff of long call (put)).
\begin{center}
\begin{tikzpicture}[declare function={func(\x) = (\x <= 2) * 0 + (\x > 2) * -(x-2);}]
\begin{axis}[domain=0:5, ymin=-3.5, ymax=1, xmax=5.5, axis y line=left, axis x line=middle,
xtick={2}, xticklabels={\(K\)}, ytick=\empty, ylabel=Payoff,
ylabel style={at={(axis description cs:0,1)}, anchor=south, rotate=-90},
xlabel=\(S_T\),
xlabel style={anchor=west}, samples=100, title=Short Call
]
\addplot[blue, thick]{func(x)};
\node[blue] () at (4.5,-1) {slope = \(-1\)};
\end{axis}
\end{tikzpicture}
\begin{tikzpicture}[declare function={func(\x) = (\x <= 2) * -(2-x) + (\x > 2) * 0;}]
\begin{axis}[domain=0:5, ymin=-3.5, ymax=1, xmax=5.5, axis y line=left, axis x line=middle,
xtick={2}, xticklabels={\(K\)}, ytick=\empty, ylabel=Payoff,
ylabel style={at={(axis description cs:0,1)}, anchor=south, rotate=-90},
xlabel=\(S_T\),
xlabel style={anchor=west}, samples=100, title=Short Put
]
\addplot[blue, thick]{func(x)};
\node[blue] () at (2,-1) {slope = 1};
\end{axis}
\end{tikzpicture}
\end{center}
\end{enumerate}
\subsection{P/L of Call and Put Options}
\begin{enumerate}
\item Since the payoff at time \(T\) of a long call/put is always nonnegative,
the (time-0) \emph{price} (or value) of the call/put option (known as (time-0)
\defn{option price} or \defn{option premium}) has to be positive
\faIcon{arrow-right} It costs \faIcon{dollar-sign} to take a long call/put.
Otherwise, a long call/put is already an arbitrage strategy --- It can possibly
make a risk-free profit without needing any cash outflow!

\begin{note}
By changing time labelling, the same is applicable to the time-\(t\) (\(t<T\))
price (or value) of the option (still expiring at time \(T\)): It has to be positive.
\end{note}

\item Notations:
\begin{itemize}
\item \defn{\(C_t\)}: call option price at time \(t\);
\item \defn{\(P_t\)}: put option price at time \(t\).
\end{itemize}
Then we have \(C_t,P_t>0\) for any time \(t<T\).

\item \label{it:lc-lp-pnl}
Given the call (put) option price at time 0: \(C_0\) (\(P_0\)), which gives the
only cash flow before time \(T\), we can determine the P/L of a long call (put)
at time \(T\):
\[
(S_T-K)_{+}-C_0e^{rT}\quad\qty((K-S_T)_{+}-P_0e^{rT}).
\]

\begin{center}
\begin{tikzpicture}[declare function={func(\x) = (\x <= 2) * -0.5 + (\x > 2) * (x-2.5);}]
\begin{axis}[domain=0:5, ymin=-1, ymax=3.5, xmax=5.5, axis y line=left, axis x line=middle,
xtick={2.5}, xticklabels={\(K+C_0e^{rT}\)}, xticklabel style={yshift=0.8cm, xshift=-0.5cm},
ytick={-0.5}, yticklabels={\(-C_0e^{rT}\)}, ylabel=P/L,
ylabel style={at={(axis description cs:0,1)}, anchor=south, rotate=-90},
xlabel=\(S_T\),
xlabel style={anchor=west}, samples=100, title=Long Call
]
\addplot[blue, thick]{func(x)};
\node[blue] () at (4.5,1) {slope = 1};
\end{axis}
\end{tikzpicture}
\begin{tikzpicture}[declare function={func(\x) = (\x <= 2) * (1.5-x) + (\x > 2) * -0.5;}]
\begin{axis}[domain=0:5, ymin=-1, ymax=3.5, xmax=5.5, axis y line=left, axis x line=middle,
xtick={1.5}, xticklabels={\(K-P_0e^{rT}\)}, xticklabel style={yshift=0.8cm, xshift=0.8cm},
ytick={-0.5}, yticklabels={\(-P_0e^{rT}\)}, ylabel=P/L,
ylabel style={at={(axis description cs:0,1)}, anchor=south, rotate=-90},
xlabel=\(S_T\),
xlabel style={anchor=west}, samples=100, title=Long Put
]
\addplot[blue, thick]{func(x)};
\node[blue] () at (1.5,1) {slope = \(-1\)};
\end{axis}
\end{tikzpicture}
\end{center}
\item \label{it:sc-sp-pnl}
Likewise, the P/L of the short call (put) at time \(T\) is the
\emph{negative} of the P/L of the long call (put) at time \(T\):
\[
-(S_T-K)_{+}+C_0e^{rT}\quad\qty(-(K-S_T)_{+}+P_0e^{rT}).
\]
\begin{note}
To take a short call (put) position (sell/write a call (put)), we
\emph{receive} the option price, not pay it. Hence, the cash \emph{outflow} at
time 0 becomes negative of the price, i.e., the time-0 cash inflow is the
price.
\end{note}
\begin{center}
\begin{tikzpicture}[declare function={func(\x) = (\x <= 2) * 0.5 + (\x > 2) * -(x-2.5);}]
\begin{axis}[domain=0:5, ymin=-3.5, ymax=1, xmax=5.5, axis y line=left, axis x line=middle,
xtick={2.5}, xticklabels={\(K+C_0e^{rT}\)}, xticklabel style={yshift=0.8cm, xshift=0.8cm},
ytick={0.5}, yticklabels={\(C_0e^{rT}\)}, ylabel=P/L,
ylabel style={at={(axis description cs:0,1)}, anchor=south, rotate=-90},
xlabel=\(S_T\),
xlabel style={anchor=west}, samples=100, title=Short Call
]
\addplot[blue, thick]{func(x)};
\node[blue] () at (4.5,-1) {slope = \(-1\)};
\end{axis}
\end{tikzpicture}
\begin{tikzpicture}[declare function={func(\x) = (\x <= 2) * -(1.5-x) + (\x > 2) * 0.5;}]
\begin{axis}[domain=0:5, ymin=-3.5, ymax=1, xmax=5.5, axis y line=left, axis x line=middle,
xtick={1.5}, xticklabels={\(K-P_0e^{rT}\)}, xticklabel style={yshift=0.8cm, xshift=-0.5cm},
ytick={0.5}, yticklabels={\(P_0e^{rT}\)}, ylabel=P/L,
ylabel style={at={(axis description cs:0,1)}, anchor=south, rotate=-90},
xlabel=\(S_T\),
xlabel style={anchor=west}, samples=100, title=Short Put
]
\addplot[blue, thick]{func(x)};
\node[blue] () at (1.5,-1) {slope = 1};
\end{axis}
\end{tikzpicture}
\end{center}
\end{enumerate}
\subsection{Moneyness of Options}
\begin{enumerate}
\item \defn{Moneyness} of an option describes whether its payoff (at time 0)
would be positive or negative if the holder were forced to exercise
the option \emph{immediately} (i.e., at the time of purchase), i.e., the cash
flow obtained after immediately exercising the option and closing out the
position.
\begin{note}
This applies to European option also even though it cannot be exercised before
expiration.
\end{note}

\item An \defn{in-the-money} (ITM)/\defn{out-of-the-money}
(OTM)/\defn{at-the-money} (ATM) option is the one that would have a
positive/negative/zero payoff if the holder were forced to exercise the
option immediately.
\begin{mnemonic}
ITM \faIcon{arrow-right} ``getting money \underline{in} our pocket \faIcon{wallet}''; OTM
\faIcon{arrow-right} ``getting money \underline{out of} our pocket \faIcon{wallet}'';
ATM \faIcon{arrow-right} ``retaining money \underline{at} our pocket \faIcon{wallet} (no more, no
less)''.
\end{mnemonic}

\item For call and put options issued (written) at time 0,
\begin{itemize}
\item call is ITM/OTM/ATM \(\iff\) \(S_0>K\) / \(S_0<K\) / \(S_0=K\) (resp.);
\item put is ITM/OTM/ATM \(\iff\) \(K>S_0\) / \(K<S_0\) / \(K=S_0\) (resp.).
\end{itemize}
\begin{note}
The time-0 payoff if exercised immediately for the call (put) is
\((S_0-K)_{+}\) (\((K-S_0)_{+}\)).
\end{note}
\end{enumerate}
\subsection{Strategies for Bullish and Bearish Speculators}
\label{subsect:bull-bear-speculate}
\begin{enumerate}
\item For speculators that are \emph{bullish} on \faIcon{apple-alt}, they can:
\begin{itemize}
\item long \faIcon{apple-alt} (buy \faIcon{apple-alt} now at spot price)

\begin{tikzpicture}[scale=0.4, declare function={func(\x) = (x-2);}]
\begin{axis}[domain=0:5, ymin=-2.5, ymax=2.5, xmax=5.5, axis y line=left, axis x line=middle,
xtick=\empty, ytick=\empty, samples=100]
\addplot[blue, thick]{func(x)};
\end{axis}
\end{tikzpicture}
\item long forward on \faIcon{apple-alt}

\begin{tikzpicture}[scale=0.4, declare function={func(\x) = (x-2);}]
\begin{axis}[domain=0:5, ymin=-2.5, ymax=2.5, xmax=5.5, axis y line=left, axis x line=middle,
xtick=\empty, ytick=\empty, samples=100]
\addplot[blue, thick]{func(x)};
\end{axis}
\end{tikzpicture}
\item long call on \faIcon{apple-alt}

\begin{tikzpicture}[scale=0.4, declare function={func(\x) = (\x <= 2) * -0.5 + (\x > 2) * (x-2.5);}]
\begin{axis}[domain=0:5, ymin=-2.5, ymax=2.5, xmax=5.5, axis y line=left, axis x line=middle,
xtick=\empty, ytick=\empty, samples=100]
\addplot[blue, thick]{func(x)};
\end{axis}
\end{tikzpicture}
\item short put on \faIcon{apple-alt}

\begin{tikzpicture}[scale=0.4, declare function={func(\x) = (\x <= 2) * -(1.5-x) + (\x > 2) * 0.5;}]
\begin{axis}[domain=0:5, ymin=-2.5, ymax=2.5, xmax=5.5, axis y line=left, axis x line=middle,
xtick=\empty, ytick=\empty, samples=100]
\addplot[blue, thick]{func(x)};
\end{axis}
\end{tikzpicture}
\end{itemize}
\begin{note}
The P/L graphs of all these positions have an ``increasing trend''.
\end{note}
\item For speculators that are \emph{bearish} on \faIcon{apple-alt}, they can:
\begin{itemize}
\item short \faIcon{apple-alt} (short sell \faIcon{apple-alt} now at spot price)

\begin{tikzpicture}[scale=0.4, declare function={func(\x) = (2-x);}]
\begin{axis}[domain=0:5, ymin=-2.5, ymax=2.5, xmax=5.5, axis y line=left, axis x line=middle,
xtick=\empty, ytick=\empty, samples=100]
\addplot[blue, thick]{func(x)};
\end{axis}
\end{tikzpicture}
\item short forward on \faIcon{apple-alt}

\begin{tikzpicture}[scale=0.4, declare function={func(\x) = (2-x);}]
\begin{axis}[domain=0:5, ymin=-2.5, ymax=2.5, xmax=5.5, axis y line=left, axis x line=middle,
xtick=\empty, ytick=\empty, samples=100]
\addplot[blue, thick]{func(x)};
\end{axis}
\end{tikzpicture}
\item short call on \faIcon{apple-alt}

\begin{tikzpicture}[scale=0.4, declare function={func(\x) = (\x <= 2) * 0.5 + (\x > 2) * -(x-2.5);}]
\begin{axis}[domain=0:5, ymin=-2.5, ymax=2.5, xmax=5.5, axis y line=left, axis x line=middle,
xtick=\empty, ytick=\empty, samples=100]
\addplot[blue, thick]{func(x)};
\end{axis}
\end{tikzpicture}
\item long put on \faIcon{apple-alt}

\begin{tikzpicture}[scale=0.4, declare function={func(\x) = (\x <= 2) * (1.5-x) + (\x > 2) * -0.5;}]
\begin{axis}[domain=0:5, ymin=-2.5, ymax=2.5, xmax=5.5, axis y line=left, axis x line=middle,
xtick=\empty, ytick=\empty, samples=100]
\addplot[blue, thick]{func(x)};
\end{axis}
\end{tikzpicture}
\end{itemize}
\begin{note}
The P/L graphs of all these positions have a ``decreasing trend''.
\end{note}
\item More option strategies are discussed in \cref{sect:option-strat}.
\end{enumerate}
