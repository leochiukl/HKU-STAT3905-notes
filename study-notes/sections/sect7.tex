\section{Multi-Period Binomial Option Pricing Model}
\label{sect:mult-period-binom}
\begin{enumerate}
\item The main weakness of one-period binomial option pricing model is that the
terminal stock price can only take \emph{two} possible values, which is
unrealistic.
\item To make the binomial option pricing model more realistic (by allowing
more possible values for the terminal stock price at least), we can utilize the
\emph{multi-period binomial option pricing model}.
\end{enumerate}
\subsection{Pricing by Backward Induction}
\label{subsect:back-induct-price}
\begin{enumerate}
\item To apply pricing by replication in \cref{sect:one-period-binom} to a
multi-period model, the main tool is the \emph{backward induction}.
\item To illustrate how backward induction works, consider the following
two-period binomial tree, which can be ``decomposed'' into three one-period
binomial trees:
\begin{center}
\begin{tikzpicture}
\node[] (s0) at (0,0) {\(S_0\)};
\node[] (su) at (4,1) {\(S_u\)};
\node[] (sd) at (4,-1) {\(S_d\)};
\node[] (suu) at (8,2) {\(S_{uu}\)};
\node[] (sud) at (8,0) {\(S_{ud}\)};
\node[] (sdd) at (8,-2) {\(S_{dd}\)};
\draw[-Latex, blue] (s0.east) -- (su.west);
\draw[-Latex, blue] (s0.east) -- (sd.west);
\draw[-Latex, violet] (su.east) -- (suu.west);
\draw[-Latex, violet] (su.east) -- (sud.west);
\draw[-Latex, orange] (sd.east) -- (sud.west);
\draw[-Latex, orange] (sd.east) -- (sdd.west);
\node[] () at (0,3) {Time 0};
\node[] () at (4,3) {Time \(h\)};
\node[] () at (8,3) {Time \(2h\)};
\end{tikzpicture}
\end{center}
\item Now, we work \emph{backward} (work from right to left), and consider the
two one-period binomial trees on the right one by one.

\item Consider first the one-period binomial tree branching out of the \(S_u\) node:
\begin{center}
\begin{tikzpicture}
\node[] (s0) at (0,0) {\(S_0\)};
\node[] (su) at (4,1) {\(S_u\)};
\node[] (sd) at (4,-1) {\(S_d\)};
\node[] (suu) at (8,2) {\(S_{uu}\)};
\node[] (sud) at (8,0) {\(S_{ud}\)};
\node[] (sdd) at (8,-2) {\(S_{dd}\)};
\draw[-Latex, blue, opacity=0.2] (s0.east) -- (su.west);
\draw[-Latex, blue, opacity=0.2] (s0.east) -- (sd.west);
\draw[-Latex, violet] (su.east) -- (suu.west);
\draw[-Latex, violet] (su.east) -- (sud.west);
\draw[-Latex, orange, opacity=0.2] (sd.east) -- (sud.west);
\draw[-Latex, orange, opacity=0.2] (sd.east) -- (sdd.west);
\node[] () at (0,3) {Time 0};
\node[] () at (4,3) {Time \(h\)};
\node[] () at (8,3) {Time \(2h\)};

\node[below = 1pt of suu, ForestGreen] (vuu) {\(V_{uu}\)};
\node[below = 1pt of sud, ForestGreen] (vud) {\(V_{ud}\)};
\node[below = 1pt of sdd, ForestGreen] (vdd) {\(V_{dd}\)};
\node[below = 1pt of su, red] (vu) {\(V_{u}\)\faIcon{question}};
%\node[below = 1pt of sd, brown] {\(V_{d}\)};

%\draw[-Latex, dashed, brown] (vuu.west) to[bend right] (vu.north east);
%\draw[-Latex, dashed, brown] (vud.west) to[bend left] (vu.south east);
%\node[font=\huge, brown] () at (6,0.5) {\faIcon[regular]{arrow-alt-circle-left}};
\end{tikzpicture}
\end{center}
\begin{note}
Here we assume the terminal payoffs (i.e., payoffs at the latest time point) of
the derivative \faIcon{scroll} at every terminal node, i.e.,
\(V_{uu},V_{ud},\text{ and }V_{dd}\) here, are well-defined and known.

This is \underline{not} the case for American options. See \cref{subsect:amer-options}.
\end{note}

With the terminal payoffs \(V_{uu}\) and \(V_{ud}\), we can use pricing by
replication (or risk-neutral pricing) for this one-period binomial tree to find
out the (time-\(h\)) value of \faIcon{scroll} at the \(S_u\) node, \(V_u\):
\[
V_u=e^{-\delta h}[V_{uu}q+V_{ud}(1-q)]
\]
where \(\displaystyle q=\frac{e^{(r-\delta)h}-d}{u-d}\).

\begin{center}
\begin{tikzpicture}
\node[] (s0) at (0,0) {\(S_0\)};
\node[] (su) at (4,1) {\(S_u\)};
\node[] (sd) at (4,-1) {\(S_d\)};
\node[] (suu) at (8,2) {\(S_{uu}\)};
\node[] (sud) at (8,0) {\(S_{ud}\)};
\node[] (sdd) at (8,-2) {\(S_{dd}\)};
\draw[-Latex, blue, opacity=0.2] (s0.east) -- (su.west);
\draw[-Latex, blue, opacity=0.2] (s0.east) -- (sd.west);
\draw[-Latex, violet] (su.east) -- (suu.west);
\draw[-Latex, violet] (su.east) -- (sud.west);
\draw[-Latex, orange, opacity=0.2] (sd.east) -- (sud.west);
\draw[-Latex, orange, opacity=0.2] (sd.east) -- (sdd.west);
\node[] () at (0,3) {Time 0};
\node[] () at (4,3) {Time \(h\)};
\node[] () at (8,3) {Time \(2h\)};

\node[below = 1pt of suu, ForestGreen] (vuu) {\(V_{uu}\)};
\node[below = 1pt of sud, ForestGreen] (vud) {\(V_{ud}\)};
\node[below = 1pt of sdd, ForestGreen] (vdd) {\(V_{dd}\)};
\node[below = 1pt of su, ForestGreen] (vu) {\(V_{u}\)\faIcon{check}};
%\node[below = 1pt of sd, brown] {\(V_{d}\)};

\draw[-Latex, dashed, brown] (vuu.west) to[bend right] (vu.north east);
\draw[-Latex, dashed, brown] (vud.west) to[bend left] (vu.south east);
\node[font=\huge, brown] () at (6,0.5) {\faIcon[regular]{arrow-alt-circle-left}};
\end{tikzpicture}
\end{center}

\begin{note}
Since the growth factors \(u\) and \(d\) are the same throughout a multi-period
binomial tree, the risk-neutral probability \(q\) is also the same throughout.
\end{note}

\item Now, we repeat this procedure for the one-period binomial tree branching
out of the \(S_d\) node:

\[
V_d=e^{-\delta h}[V_{ud}q+V_{dd}(1-q)].
\]

\begin{center}
\begin{tikzpicture}
\node[] (s0) at (0,0) {\(S_0\)};
\node[] (su) at (4,1) {\(S_u\)};
\node[] (sd) at (4,-1) {\(S_d\)};
\node[] (suu) at (8,2) {\(S_{uu}\)};
\node[] (sud) at (8,0) {\(S_{ud}\)};
\node[] (sdd) at (8,-2) {\(S_{dd}\)};
\draw[-Latex, blue, opacity=0.2] (s0.east) -- (su.west);
\draw[-Latex, blue, opacity=0.2] (s0.east) -- (sd.west);
\draw[-Latex, violet, opacity=0.2] (su.east) -- (suu.west);
\draw[-Latex, violet, opacity=0.2] (su.east) -- (sud.west);
\draw[-Latex, orange] (sd.east) -- (sud.west);
\draw[-Latex, orange] (sd.east) -- (sdd.west);
\node[] () at (0,3) {Time 0};
\node[] () at (4,3) {Time \(h\)};
\node[] () at (8,3) {Time \(2h\)};

\node[below = 1pt of suu, ForestGreen] (vuu) {\(V_{uu}\)};
\node[below = 1pt of sud, ForestGreen] (vud) {\(V_{ud}\)};
\node[below = 1pt of sdd, ForestGreen] (vdd) {\(V_{dd}\)};
\node[below = 1pt of su, ForestGreen] (vu) {\(V_{u}\)};
\node[below = 1pt of sd, ForestGreen] (vd) {\(V_{d}\)\faIcon{check}};
%\node[below = 1pt of sd, brown] {\(V_{d}\)};

\draw[-Latex, dashed, brown] (vdd.west) to[bend left] (vd.south east);
\draw[-Latex, dashed, brown] (vud.west) to[bend right] (vd.north east);
\node[font=\huge, brown] () at (6,-1.5) {\faIcon[regular]{arrow-alt-circle-left}};
\end{tikzpicture}
\end{center}

\item Lastly, repeat this procedure again for the one-period binomial tree
branching out of the \(S_0\) node:
\[
V_0=e^{-\delta h}[V_{u}q+V_{d}(1-q)],
\]
and we have found the time-0 value of \faIcon{scroll}: \(V_0\)!
\begin{center}
\begin{tikzpicture}
\node[] (s0) at (0,0) {\(S_0\)};
\node[] (su) at (4,1) {\(S_u\)};
\node[] (sd) at (4,-1) {\(S_d\)};
\node[] (suu) at (8,2) {\(S_{uu}\)};
\node[] (sud) at (8,0) {\(S_{ud}\)};
\node[] (sdd) at (8,-2) {\(S_{dd}\)};
\draw[-Latex, blue] (s0.east) -- (su.west);
\draw[-Latex, blue] (s0.east) -- (sd.west);
\draw[-Latex, violet, opacity=0.2] (su.east) -- (suu.west);
\draw[-Latex, violet, opacity=0.2] (su.east) -- (sud.west);
\draw[-Latex, orange, opacity=0.2] (sd.east) -- (sud.west);
\draw[-Latex, orange, opacity=0.2] (sd.east) -- (sdd.west);
\node[] () at (0,3) {Time 0};
\node[] () at (4,3) {Time \(h\)};
\node[] () at (8,3) {Time \(2h\)};

\node[below = 1pt of suu, ForestGreen] (vuu) {\(V_{uu}\)};
\node[below = 1pt of sud, ForestGreen] (vud) {\(V_{ud}\)};
\node[below = 1pt of sdd, ForestGreen] (vdd) {\(V_{dd}\)};
\node[below = 1pt of su, ForestGreen] (vu) {\(V_{u}\)};
\node[below = 1pt of sd, ForestGreen] (vd) {\(V_{d}\)};
\node[below = 1pt of s0, ForestGreen] (v0) {\(V_{0}\)\faIcon{check}};

\draw[-Latex, dashed, brown] (vd.west) to[bend left] (v0.south east);
\draw[-Latex, dashed, brown] (vu.west) to[bend right] (v0.north east);
\node[font=\huge, brown] () at (2,-0.5) {\faIcon[regular]{arrow-alt-circle-left}};
\end{tikzpicture}
\end{center}

\item Similar approach would work for a general \(n\)-period binomial tree, and
it is known as \defn{backward induction}.
\end{enumerate}
\subsection{Risk-Neutral Pricing}
\begin{enumerate}
\item As we can see, if the number of periods \(n\) involved in the tree is
large, the backward induction approach would require many computations and
would be rather cumbersome.

\item Fortunately, for \emph{European option} (or any derivative with a
well-defined ``terminal'' time point and known ``terminal'' payoffs) it turns out
that there is an alternative and more convenient formula to find out its time-0
value, which is again somehow related to the idea of \emph{risk-neutral
pricing}.

\item To get the formula for the two-period tree, consider:
\begin{align*}
V_0&=e^{-\delta h}[{\color{blue}V_u}q+{\color{violet}V_d}(1-q)]\\
&=e^{-\delta h}\qty{{\color{blue}e^{-\delta h}[V_{uu}q+V_{ud}(1-q)]}q+{\color{violet}e^{-\delta h}[V_{ud}q+V_{dd}(1-q)]}(1-q)} \\
&=e^{-\delta (2h)}\qty[V_{uu}q^2+V_{ud}(2q(1-q))+V_{dd}(1-q)^2].
\end{align*}
We can note that this formula is of the ``risk-neutral pricing'' form:
\[
e^{-\delta(\text{total duration})}\expvq{\text{terminal payoff}},
\]
by identifying that the total duration is \(2h\), and the terminal payoff is
\[
\begin{cases}
V_{uu}&\text{with probability \(q\)};\\
V_{ud}&\text{with probability \(2q(1-q)\)};\\
V_{dd}&\text{with probability \((1-q)^{2}\)},\\
\end{cases}
\]
where the probabilities here are \emph{risk-neutral}.

\item \label{it:back-induct-n-period-fmla}
For a general \(n\)-period binomial tree, the time-0 value is
\[
V_0=e^{-\delta(nh)}\qty[V_{u^n}q^n+V_{u^{n-1}d}\binom{n}{1}q^{n-1}(1-q)+\dotsb+V_{d^{n}}(1-q)^{n}]
=e^{-\delta(nh)}\sum_{i=0}^{n}V_{u^{n-i}d^{i}}\binom{n}{i}q^{n-i}(1-q)^{i}.
\]
\begin{pf}
By observing the back induction process, we can see that the time-0 value has
the form of
\begin{align*}
&e^{-\delta(nh)}\sum_{\text{every terminal node}}^{}
\text{payoff at the node}\times\text{no.\ of paths to the node} \\
&\hspace{3.5cm}\times\text{product of risk-neutral probabilities arising from the backward induction process}.
\end{align*}

In an \(n\)-period binomial tree, for any \(i=0,1,\dotsc,n\), to reach the
\(S_{u^{n-i}d^{i}}\) node from the start (\(S_0\) node), \(i\) ``down'' moves
(and \(n-i\) ``up'' moves) are needed.

To get a path satisfying this requirement, we need to choose \(i\) out of \(n\)
periods to have ``down'' move \faIcon{arrow-right} \(\binom{n}{i}\) distinct
paths satisfying the requirement (the order of choosing does not matter).

Now note that for each of those paths, the product of the risk-neutral
probabilities arising from the back induction process is \(q^{n-i}(1-q)^{i}\)
(one ``up'' (``down'') move \faIcon{arrows-alt-h} one \(q\) (\(1-q\)) gets
multiplied).
\end{pf}

\item From the formula in \labelcref{it:back-induct-n-period-fmla}, we can
identify the term \(\binom{n}{i}q^{n-i}(1-q)^{i}\) as the \emph{risk-neutral
probability} of reaching the \(S_{u^{n-i}d^{i}}\) node (i.e., having \(i\)
``down'' moves) \faIcon{arrow-right} no.\ of ``up'' moves follows
the \emph{binomial distribution} (\(n\) independent trials with ``up'' probability
\(q\)) in a risk-neutral world.

\item \label{it:mult-period-rn-pricing-fmla}
After this identification (of ``appropriate'' probabilities of reaching
different nodes in a risk-neutral world), we can express the formula in
\labelcref{it:back-induct-n-period-fmla} the risk-neutral pricing form:
\begin{equation}
\label{eq:mult-period-rn-pricing-fmla}
e^{-\delta(nh)}\expvq{\text{terminal payoff}}.
\end{equation}
\end{enumerate}
\subsection{American Options}
\label{subsect:amer-options}
\begin{enumerate} 
\item For an \emph{American option}, since \defn{early exercise} (i.e.,
exercising before maturity) is possible, the option can ``end'' early
\faIcon{arrow-right} the notion of ``terminal'' is ill-defined (there is not a
unique ``terminal point''!) \faIcon{arrow-right} the formula in
\labelcref{it:rn-pricing-fmla-mult-period} is \emph{not applicable}.

\item To price an American option, we can use again \emph{backward
induction}, but with some ``twist'' not mentioned in
\cref{subsect:back-induct-price}.

\begin{remark}
\item This shows an advantage of the binomial option pricing model: It can accommodate
\emph{American options}, which are less mathematically tractable than European
options.
\item Here we shall assume that the American option can only be exercised at the
time points specified in the (\(n\)-period) binomial tree, i.e., time
\(0,h,2h,\dotsc,nh\).
\end{remark}

\item The ``twist'' is that we need to now more consciously consider the
\emph{rationality} assumption mentioned in
\labelcref{it:option-holder-rational} during the backward induction process.

\item We again consider the following two-period binomial tree as an example,
and we now work with an American option:
\begin{center}
\begin{tikzpicture}
\node[] (s0) at (0,0) {\(S_0\)};
\node[] (su) at (4,1) {\(S_u\)};
\node[] (sd) at (4,-1) {\(S_d\)};
\node[] (suu) at (8,2) {\(S_{uu}\)};
\node[] (sud) at (8,0) {\(S_{ud}\)};
\node[] (sdd) at (8,-2) {\(S_{dd}\)};
\draw[-Latex, blue] (s0.east) -- (su.west);
\draw[-Latex, blue] (s0.east) -- (sd.west);
\draw[-Latex, violet] (su.east) -- (suu.west);
\draw[-Latex, violet] (su.east) -- (sud.west);
\draw[-Latex, orange] (sd.east) -- (sud.west);
\draw[-Latex, orange] (sd.east) -- (sdd.west);
\node[] () at (0,3) {Time 0};
\node[] () at (4,3) {Time \(h\)};
\node[] () at (8,3) {Time \(2h\)};
\end{tikzpicture}
\end{center}

\item \label{it:amer-option-rational}
Consider the one-period binomial tree branching out of the \(S_u\) node:
\begin{center}
\begin{tikzpicture}
\node[] (s0) at (0,0) {\(S_0\)};
\node[] (su) at (4,1) {\(S_u\)};
\node[] (sd) at (4,-1) {\(S_d\)};
\node[] (suu) at (8,2) {\(S_{uu}\)};
\node[] (sud) at (8,0) {\(S_{ud}\)};
\node[] (sdd) at (8,-2) {\(S_{dd}\)};
\draw[-Latex, blue, opacity=0.2] (s0.east) -- (su.west);
\draw[-Latex, blue, opacity=0.2] (s0.east) -- (sd.west);
\draw[-Latex, violet] (su.east) -- (suu.west);
\draw[-Latex, violet] (su.east) -- (sud.west);
\draw[-Latex, orange, opacity=0.2] (sd.east) -- (sud.west);
\draw[-Latex, orange, opacity=0.2] (sd.east) -- (sdd.west);
\node[] () at (0,3) {Time 0};
\node[] () at (4,3) {Time \(h\)};
\node[] () at (8,3) {Time \(2h\)};

\node[below = 1pt of suu, ForestGreen] (vuu) {\(V_{uu}\)};
\node[below = 1pt of sud, ForestGreen] (vud) {\(V_{ud}\)};
\node[below = 1pt of sdd, ForestGreen] (vdd) {\(V_{dd}\)};
\node[below = 1pt of su, red] (vu) {\(V_{u}\)\faIcon{question}};

\node[Maroon] () at (4,2) {early exercise?};
\node[Maroon] () at (4,1.5) {hold \faIcon{scroll}?};
\end{tikzpicture}
\end{center}

At the \(S_u\) node, the option holder has an opportunity to choose whether to
early exercise or not, i.e., choosing between early exercise and \emph{holding}
\faIcon{scroll} (at least) until the next time point in the tree.

By rationality, the holder would choose the one with ``more positive''
\emph{value}. The values for ``early exercise''
and ``holding'' are called \defn{immediate exercise value} and \defn{holding
value} respectively, which are given by:

\begin{itemize}
\item \underline{immediate exercise value}: perform ``early exercise''
\faIcon{arrow-right} the American option ``is reduced to'' an European option
expiring at current time point \faIcon{arrow-right} immediate exercise value =
\((S_u-K)_{+}\) (\((K-S_u)_{+}\)) for call (put);

\item \underline{holding value}: perform ``holding'' \faIcon{arrow-right} the
American option ``is reduced to'' a derivative with payoffs \(V_{uu}\) after
``up'' and \(V_{ud}\) after ``down'' in one-period tree\footnote{The derivative
is not going to ``end'' until the termination time point of the tree.}
\faIcon{arrow-right} holding value = value of that derivative  (which can be
found by pricing by replication/risk-neutral pricing).
\end{itemize}

\begin{note}
In case both values are identical, we assume the holder would \emph{not} early exercise.
\end{note}


\item \label{it:amer-option-value}
Consequently, when the immediate exercise value (holding value) is
higher, the American option would be ``reduced to'' an currently expiring
European option (a derivative with payoffs \(V_{uu}\) after ``up'' and
\(V_{ud}\) after ``down'' in one-period tree), which again has the value of
immediate exercise value (holding value).

In other words, the value of the American option at the \(S_u\) node (or any
other non-rightmost node) is the \emph{maximum} of the holding value and immediate exercise
value.

\item Suppose the holding value is higher in this case, and we have:
\[
V_u=e^{-\delta h}[V_{uu}q+V_{ud}(1-q)]
\]
\begin{center}
\begin{tikzpicture}
\node[] (s0) at (0,0) {\(S_0\)};
\node[] (su) at (4,1) {\(S_u\)};
\node[] (sd) at (4,-1) {\(S_d\)};
\node[] (suu) at (8,2) {\(S_{uu}\)};
\node[] (sud) at (8,0) {\(S_{ud}\)};
\node[] (sdd) at (8,-2) {\(S_{dd}\)};
\draw[-Latex, blue, opacity=0.2] (s0.east) -- (su.west);
\draw[-Latex, blue, opacity=0.2] (s0.east) -- (sd.west);
\draw[-Latex, violet] (su.east) -- (suu.west);
\draw[-Latex, violet] (su.east) -- (sud.west);
\draw[-Latex, orange, opacity=0.2] (sd.east) -- (sud.west);
\draw[-Latex, orange, opacity=0.2] (sd.east) -- (sdd.west);
\node[] () at (0,3) {Time 0};
\node[] () at (4,3) {Time \(h\)};
\node[] () at (8,3) {Time \(2h\)};

\node[below = 1pt of suu, ForestGreen] (vuu) {\(V_{uu}\)};
\node[below = 1pt of sud, ForestGreen] (vud) {\(V_{ud}\)};
\node[below = 1pt of sdd, ForestGreen] (vdd) {\(V_{dd}\)};
\node[below = 1pt of su, ForestGreen] (vu) {\(V_{u}\)\faIcon{check}};

\node[Maroon] () at (4,1.5) {hold \faIcon{scroll}\faIcon{check}}; \draw[-Latex, dashed, brown] (vuu.west) to[bend right] (vu.north east);
\draw[-Latex, dashed, brown] (vud.west) to[bend left] (vu.south east);
\node[font=\huge, brown] () at (6,0.5) {\faIcon[regular]{arrow-alt-circle-left}};
\end{tikzpicture}
\end{center}

\item Now, we repeat this procedure for the one-period binomial tree branching
out of the \(S_d\) node, and suppose the \emph{immediate exercise value} is
higher:
\[
V_d^*=\begin{cases}
(S_d-K)_{+}\text{ for call}; \\
(K-S_d)_{+}\text{ for put}. \\
\end{cases}
\]
\begin{note}
To signify that the immediate exercise value is higher, we conventionally add
an asterisk \(*\) as a superscript on the value of American option. This still
applies when we write the actual number, e.g. we write \(100^*\) rather than
\(100\) (but of course the actual value represented does not change).
\end{note}
\begin{center}
\begin{tikzpicture}
\node[] (s0) at (0,0) {\(S_0\)};
\node[] (su) at (4,1) {\(S_u\)};
\node[] (sd) at (4,-1) {\(S_d\)};
\node[] (suu) at (8,2) {\(S_{uu}\)};
\node[] (sud) at (8,0) {\(S_{ud}\)};
\node[] (sdd) at (8,-2) {\(S_{dd}\)};
\draw[-Latex, blue, opacity=0.2] (s0.east) -- (su.west);
\draw[-Latex, blue, opacity=0.2] (s0.east) -- (sd.west);
\draw[-Latex, violet, opacity=0.2] (su.east) -- (suu.west);
\draw[-Latex, violet, opacity=0.2] (su.east) -- (sud.west);
\draw[-Latex, orange] (sd.east) -- (sud.west);
\draw[-Latex, orange] (sd.east) -- (sdd.west);
\node[] () at (0,3) {Time 0};
\node[] () at (4,3) {Time \(h\)};
\node[] () at (8,3) {Time \(2h\)};

\node[below = 1pt of suu, ForestGreen] (vuu) {\(V_{uu}\)};
\node[below = 1pt of sud, ForestGreen] (vud) {\(V_{ud}\)};
\node[below = 1pt of sdd, ForestGreen] (vdd) {\(V_{dd}\)};
\node[below = 1pt of su, ForestGreen] (vu) {\(V_{u}\)};
\node[below = 1pt of sd, ForestGreen] (vd) {\(V_{d}^*\)\faIcon{check}};

\node[Maroon] () at (4,-2) {early exercise \faIcon{check}};
\node[font=\huge, brown] () at (6,-1.5) {\faIcon[regular]{arrow-alt-circle-left}};
\end{tikzpicture}
\end{center}

\item Lastly, repeat this procedure again for the one-period binomial tree
branching out of the \(S_0\) node, and suppose the holding value is higher:
\[
V_0=e^{-\delta h}[V_{u}q+V_{d}^{{\color{magenta}*}}(1-q)].
\]
\begin{note}
We use \(V_d^*\) in the formula since it is indeed the value (or payoff) of
American option at node \(S_d\).
\end{note}

\begin{center}
\begin{tikzpicture}
\node[] (s0) at (0,0) {\(S_0\)};
\node[] (su) at (4,1) {\(S_u\)};
\node[] (sd) at (4,-1) {\(S_d\)};
\node[] (suu) at (8,2) {\(S_{uu}\)};
\node[] (sud) at (8,0) {\(S_{ud}\)};
\node[] (sdd) at (8,-2) {\(S_{dd}\)};
\draw[-Latex, blue] (s0.east) -- (su.west);
\draw[-Latex, blue] (s0.east) -- (sd.west);
\draw[-Latex, violet, opacity=0.2] (su.east) -- (suu.west);
\draw[-Latex, violet, opacity=0.2] (su.east) -- (sud.west);
\draw[-Latex, orange, opacity=0.2] (sd.east) -- (sud.west);
\draw[-Latex, orange, opacity=0.2] (sd.east) -- (sdd.west);
\node[] () at (0,3) {Time 0};
\node[] () at (4,3) {Time \(h\)};
\node[] () at (8,3) {Time \(2h\)};

\node[below = 1pt of suu, ForestGreen] (vuu) {\(V_{uu}\)};
\node[below = 1pt of sud, ForestGreen] (vud) {\(V_{ud}\)};
\node[below = 1pt of sdd, ForestGreen] (vdd) {\(V_{dd}\)};
\node[below = 1pt of su, ForestGreen] (vu) {\(V_{u}\)};
\node[below = 1pt of sd, ForestGreen] (vd) {\(V_{d}^{*}\)};
\node[below = 1pt of s0, ForestGreen] (v0) {\(V_{0}\)\faIcon{check}};

\draw[-Latex, dashed, brown] (vd.west) to[bend left] (v0.south east);
\draw[-Latex, dashed, brown] (vu.west) to[bend right] (v0.north east);
\node[font=\huge, brown] () at (2,-0.5) {\faIcon[regular]{arrow-alt-circle-left}};
\end{tikzpicture}
\end{center}

\item An important shortcut regarding American option is suggested by the result below:
\begin{theorem}
\label{thm:amer-call-no-div-no-early-ex}
It is \emph{never} rational to early exercise an \emph{American call} on an
asset \faIcon{apple-alt} with no dividends etc., when \(r\ge 0\) (typically the
case).
\end{theorem}
\begin{pf}
Consider an European call and an European put with the same terms (except the
exercise style of course) as the American call (same underlying asset, strike
price, and expiration date). Let \(C_t^{E}\) and \(P_t^{E}\) be their time-\(t\) prices
respectively, and let \(C_t^{A}\) be the time-\(t\) price of the American call.

For any \(t\in[0,T)\), by (non-generalized) put-call parity
(\cref{thm:put-call-parity}) (with changes in time labelling if necessary), we
have
\[
C_t^{E}+Ke^{-r(T-t)}=P_t^{E}+S_t.
\]

Firstly, note that we must have \(C_t^A\ge C_t^E\). (Otherwise, longing the
American call and shorting the European call at time \(t\) would lead to
arbitrage\footnote{For such strategy, the total cash flow at time \(t\) is
positive. Next, we do not early exercise our American call, and at time \(T\),
we exercise our American call if the European call is exercised
\faIcon{arrow-right} zero total cash flow.}.) Since the European call price
\(C_t^E\) is always positive, it means the American call price \(C_t^A\) is
also positive.

Consequently, when \(S_t\le K\) (where early exercise is of zero value), since
\emph{selling} the American call (to receive a \emph{positive} cash flow) is of
positive value, it is not rational to early exercise.

Now consider the case where \(S_t>K\). Note that
\[
C_t^{A}\ge C_t^{E}>C_t^{E}-P_t^{E}=S_t-Ke^{-rT}\ge S_t-K,
\]
which means \emph{selling} the American call (to receive the price \(C_t^A\))
has a higher value than early exercise \faIcon{arrow-right} early exercise is
again not rational.
\end{pf}

\begin{note}
Because of this result, such American call can be readily ``reduced to'' the
\emph{European call} with the same terms (except exercise style), and methods
for European calls can still be used, e.g. risk-neutral pricing based on
terminal payoff (\cref{eq:mult-period-rn-pricing-fmla}).
\end{note}
\end{enumerate}
