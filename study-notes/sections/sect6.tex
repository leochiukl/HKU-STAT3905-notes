\section{One-Period Binomial Option Pricing Model}
\label{sect:one-period-binom}
\begin{enumerate}
\item In \cref{sect:options,sect:option-strat}, we discuss various aspects of
options, but so far we do not have much idea about how to actually \emph{price}
the options and option strategies.

\item Starting from here, we shall discuss some models for pricing them (which
is not so straightforward!).
\end{enumerate}

\subsection{Binomial Option Pricing Model}
\begin{enumerate}
\item In a \emph{binomial option pricing model}, we are assumed to be in a
perfect market having (but not limited to) the following two assets:
\begin{itemize}
\item a risky stock \faIcon{apple-alt} which pays dividend continuously at a
dividend yield \(\delta\) (if \(\delta=0\), it means there is no
dividend)\footnote{Following the previous convention, we assume the continuous
dividends are reinvested into the stock \faIcon{apple-alt} automatically. So if
we own \(N_0\) shares of \faIcon{apple-alt} at time 0, the number of shares we
own would accumulate to \(N_0 e^{\delta t}\) at time \(t\).};

\item a (risk-free zero-coupon) bond \faIcon{file-invoice-dollar} with an
annual continuously compounded risk-free rate \(r\)
\end{itemize}
\begin{remark}
\item Of course the no-arbitrage principle is still assumed. (It is assumed
throughout this notes unless stated otherwise.)
\item Recall that the risk-free rate for lending and borrowing are identical in
a perfect market.
\item Following \labelcref{it:buy-sell-any-num}, we shall assume we can freely
buy/sell any amount of the stock \faIcon{apple-alt} and bond
\faIcon{file-invoice-dollar} at spot price. (Freely buying/selling any amount
of bond means that we can lend/sell any amount of money risk-free.)
\end{remark}

\item In a \defn{one-period binomial option pricing model} whose duration is
\(h>0\) (years), the spot price of the stock \faIcon{apple-alt} at the end of
the (only one) period (time \(h\)) is assumed to take exactly \emph{two}
distinct possible values (hence
``binomial''):
\[
S_h=
\begin{cases}
\text{\defn{\(S_u\)}}\triangleq S_0 u&\text{with probability \(p\)};\\
\text{\defn{\(S_d\)}}\triangleq S_0 d&\text{with probability \(1-p\)}.\\
\end{cases}
\]
where \(u>d>0\), and \(p\in (0,1)\)\footnote{The ``up'' probability can be
neither 0 nor 1, so that the stock is indeed ``risky''.}.

\begin{remark}
\item The values ``\(u\)'' and ``\(d\)'' are sometimes called \defn{growth
factors}.  \item Here ``\(u\)'' and ``\(d\)'' may be interpreted as suggesting
``up'' and ``down'' respectively. But one should be careful that ``down'' here
may not mean the price goes down, i.e., \(S_d\) is lower than \(S_0\): It turns
out that it is possible for \(d\ge 1\).
\end{remark}

\item \defn{Binomial tree} representation of the one-period model:
\begin{center}
\begin{tikzpicture}
\node[] (s0) at (0,0) {\(S_0\)};
\node[] (su) at (4,1) {\(S_u=S_0u\)};
\node[] (sd) at (4,-1) {\(S_d=S_0d\)};
\draw[-Latex] (s0.east) -- (su.west)
node[midway, above=0.3cm] {``up''};
\draw[-Latex] (s0.east) -- (sd.west)
node[midway, below=0.3cm] {``down''};
\node[] () at (0,2) {Time 0};
\node[] () at (4,2) {Time \(h\)};
\end{tikzpicture}
\end{center}

\item For a \defn{multi-period binomial option pricing model} (discussed in
details in \cref{sect:mult-period-binom}) with \(n\) periods (or \defn{\(n\)-period
binomial option pricing model}),
\begin{itemize}
\item the duration of \emph{each} period is a constant \(h>0\);
\item the same growth factors \(u\) and \(d\) and ``up'' probability \(p\) are
used in \emph{each} period, and they are applied on the spot price at the
\emph{beginning} of the period.
\end{itemize}

\item Binomial tree representation of a two-period model:
\begin{center}
\begin{tikzpicture}
\node[] (s0) at (0,0) {\(S_0\)};
\node[] (su) at (4,1) {\(S_u=S_0u\)};
\node[] (sd) at (4,-1) {\(S_d=S_0d\)};
\node[] (suu) at (8,2) {\(S_{uu}=S_0uu\)};
\node[] (sud) at (8,0) {\(S_{ud}=S_0ud\)};
\node[] (sdd) at (8,-2) {\(S_{dd}=S_0dd\)};
\draw[-Latex] (s0.east) -- (su.west);
\draw[-Latex] (s0.east) -- (sd.west);
\draw[-Latex] (su.east) -- (suu.west);
\draw[-Latex] (su.east) -- (sud.west);
\draw[-Latex] (sd.east) -- (sud.west);
\draw[-Latex] (sd.east) -- (sdd.west);
\node[] () at (0,3) {Time 0};
\node[] () at (4,3) {Time \(h\)};
\node[] () at (8,3) {Time \(2h\)};
\end{tikzpicture}
\end{center}
\begin{remark}
\item The number of ``\(u\)'' and ``\(d\)'' in the subscript of ``\(S\)''
suggests the number of ``up'' and ``down'' moves (resp.) needed to get to that
price (the order of the letters is not important).  Sometimes to make the
notation more compact, we may express the subscript using power, e.g.
\(S_{u^8d^2}\).

\item The stock price at the latest time point in a binomial option pricing
model is known as \defn{terminal stock price}. For a \(n\)-period binomial
option pricing model, there are \(n+1\) distinct possible terminal stock
prices.\footnote{This is because the subscript of ``\(S\)'' can contain
\(0,1,\dotsc,n\) ``\(u\)''s (with the rest being ``\(d\)''s)
\faIcon{arrow-right} \(n+1\) different subscripts.}

\item As we can see, a multi-period binomial tree can actually be regarded as a
combination of multiple one-period binomial trees. Through this connection,
methods developed for one-period binomial option pricing model (in
\cref{sect:one-period-binom}) can still be utilized for the multi-period
binomial option pricing model (in \cref{sect:mult-period-binom}).
\end{remark}
\item In \cref{sect:one-period-binom}, we focus on one-period binomial option
pricing model.
\end{enumerate}
\subsection{Pricing by Replication}
\begin{enumerate}
\item Consider a derivative \faIcon{scroll} on the stock \faIcon{apple-alt}
(whose value is derived from the price of \faIcon{apple-alt}) that matures at
time \(h\), and let \(V_t\) be its value at time \(t\).

\item Write its payoff (at time \(h\)) as
\[
V_h
=
\begin{cases}
V_u&\text{if }S_h=S_0u;\\
V_d&\text{if }S_h=S_0d.\\
\end{cases}
\]
\begin{center}
\begin{tikzpicture}
\node[] (s0) at (0,0) {\(S_0\)};
\node[] (su) at (4,1) {\(S_u=S_0u\)};
\node[] (sd) at (4,-1) {\(S_d=S_0d\)};
\draw[-Latex] (s0.east) -- (su.west);
\draw[-Latex] (s0.east) -- (sd.west);
\node[] () at (0,2) {Time 0};
\node[] () at (4,2) {Time \(h\)};
\node[] () at (6,2) {Payoff};
\node[] () at (6,1) {\(V_u\)};
\node[] () at (6,-1) {\(V_d\)};
\end{tikzpicture}
\end{center}

\item We are interested in finding the time-0 price/value of the derivative
\faIcon{scroll}: \(V_0\). One approach for pricing the derivative is called
\defn{pricing by replication}, which involves \emph{replicating} the payoff of
\faIcon{scroll} using just the stock \faIcon{apple-alt} and bond
\faIcon{file-invoice-dollar} (as we know the prices of these ``basic'' assets).

\item At time 0, we consider a portfolio consisting of \(\Delta\) shares of
stock \faIcon{apple-alt} and an amount \(B\) in risk-free bond
\faIcon{file-invoice-dollar} (lending \(B\)).

\begin{note}
Negative \(\Delta\) or \(B\) indicates a short position (with absolute value of
that amount of stock/in bond), just like the convention for ``owning/lending a
negative amount''.
\end{note}

\item Its time-0 value is \(\Delta S_0+B\) and its payoff (at time \(h\)) is
\[
\begin{cases}
\Delta e^{\delta h}S_0u+Be^{rh}&\text{if }S_h=S_0u;\\
\Delta e^{\delta h}S_0d+Be^{rh}&\text{if }S_h=S_0d.\\
\end{cases}
\]

\item \label{it:rep-port}
To \emph{replicate} the payoff of \faIcon{scroll}, we set the following
system of linear equations (in variables \(\Delta\) and \(B\)):
\[
\begin{cases}
\Delta e^{\delta h}S_0u+Be^{rh}=V_u\\
\Delta e^{\delta h}S_0d+Be^{rh}=V_d
\end{cases}.
\]
Since \(u\ne d\), this system can be solved uniquely:

\begin{equation}
\label{eq:rep-port}
\Delta=e^{-\delta h}\frac{V_u-V_d}{S_0(u-d)},\quad B=e^{-rh}\frac{uV_d-dV_u}{u-d}.
\end{equation}

The portfolio with these values of \(\Delta\) and \(B\) is called
\defn{replicating portfolio} of the derivative \faIcon{scroll}.

\item \label{it:price-by-rep-fmla}
By the law of one price, since the replicating portfolio and the
derivative \faIcon{scroll} have the same payoff, they have the same time-0 value:
\[
V_0=\Delta S_0+B,
\]
where \(\Delta\) and \(B\) are given in \cref{eq:rep-port}. This gives the
pricing formula for the derivative \faIcon{scroll}.

\item \label{it:price-by-rep-fmla-exp-form}
The pricing formula in \labelcref{it:price-by-rep-fmla} can be rewritten
in the following ``discounted expectation'' form:
\begin{align*}
V_0&=\Delta S_0+B\\
&=e^{-\delta h}S_0\frac{V_u-V_d}{S_0(u-d)}+e^{-rh}\frac{uV_d-dV_u}{u-d}\\
&=e^{-rh}\frac{e^{(r-\delta)h}(V_u-V_d)+uV_d-dV_u}{u-d}\\
&=e^{-rh}\qty[V_uq+V_d(1-q)],
\end{align*}
where \(\displaystyle q=\frac{e^{(r-\delta)h}-d}{u-d}\). 

If we ``regard'' \(q\) as the ``up'' probability (but of course \(q\) is not
the \emph{true} ``up'' probability), then the expression is simply the
\emph{discounted expected payoff}\footnote{or expected discounted payoff
(\emph{expected present value}/\emph{actuarial present value} of payoff) if we
expand the terms \faIcon{arrow-right} \(V_ue^{-rh}q+V_de^{-rh}(1-q)\)} of the
derivative \faIcon{scroll} at the risk-free rate. More details are discussed in
\cref{subsect:rn-pricing}.

\item Another way to develop the formula in \labelcref{it:price-by-rep-fmla} is
to consider another kind of replication: replicating the payoff of a position
in \emph{bond} \faIcon{file-invoice-dollar} (i.e., a constant payoff) using the
derivative \faIcon{scroll} and the stock \faIcon{apple-alt} instead.

\item We consider a portfolio consisting of a \emph{short} position in the
derivative \faIcon{scroll} and \(\Delta\) shares of stock \faIcon{apple-alt}.
\begin{intuition}
We consider a short position in \faIcon{scroll} since from the approach above
we know, in payoff,
\[
\text{\(\Delta\) shares of \faIcon{apple-alt}} + \text{\(B\) in
\faIcon{file-invoice-dollar}} = \text{1 \faIcon{scroll}}.
\]
``Rearranging'' it gives
\[
\text{\(\Delta\) shares of \faIcon{apple-alt}} \underbrace{- \text{1 \faIcon{scroll}}}_{\text{short position}}
= -\text{\(B\) in \faIcon{file-invoice-dollar}}.
\]
\end{intuition}

\item Its time-0 value is \(\Delta S_0-V_0\), and its payoff (at time \(h\)) is
\[
\begin{cases}
\Delta e^{\delta h}S_0u-V_u&\text{if }S_h=S_0u;\\
\Delta e^{\delta h}S_0d-V_d&\text{if }S_h=S_0d.\\
\end{cases}
\]
To replicate a \emph{constant} payoff, we need to have
\begin{equation}
\label{eq:short-deriv-stock-rep}
\Delta e^{\delta h}S_0u-V_u=\Delta e^{\delta h}S_0d-V_d.
\end{equation}
Rearranging it gives
\[
\Delta=e^{-\delta h}\frac{V_u-V_d}{S_0u-S_0d}.
\]
So, the portfolio consisting of \(\displaystyle \Delta=e^{-\delta
h}\frac{V_u-V_d}{S_0u-S_0d}\) shares of \faIcon{apple-alt} and a short position
in \faIcon{scroll} has also a constant payoff, which is the common value
in \cref{eq:short-deriv-stock-rep}, i.e.,
\[
V=\Delta e^{\delta h}S_0u-V_u
=e^{-\delta h}\frac{V_u-V_d}{S_0u-S_0d}e^{\delta h}S_0u-V_u
=\frac{uV_u-uV_d-uV_u+dV_u}{u-d}
=\frac{dV_u-uV_d}{u-d}.
\]

\item Note that the time-0 value of the position in bond with constant payoff
\(V\) at time \(h\) is \(Ve^{-rh}\) (by no-arbitrage principle). Hence, by law
of one price, \(\Delta S_0-V_0=Ve^{-rh}\), which implies
\begin{align*}
V_0&=e^{-\delta h}\frac{V_u-V_d}{S_0u-S_0d}S_0-\frac{dV_u-uV_d}{u-d}e^{-rh} \\
&=e^{-rh}\frac{e^{(r-\delta)h}(V_u-V_d)+uV_d-dV_u}{u-d} \\
&=e^{-rh}\qty[V_uq+V_d(1-q)],
\end{align*}
which is identical to the formula in \labelcref{it:price-by-rep-fmla-exp-form} (or
\labelcref{it:price-by-rep-fmla}).
\end{enumerate}
\subsection{Risk-Neutral Pricing}
\label{subsect:rn-pricing}
\begin{enumerate}
\item As suggested in \labelcref{it:price-by-rep-fmla-exp-form}, it turns out
that the formula for pricing by replication can be expressed in a ``discounted
expectation'' form. Such formula is indeed known as the \defn{risk-neutral
pricing formula}, and \(q\) is the \defn{risk-neutral probability} of ``up''.
\item Before proceeding further, let us first justify that \(q\) can indeed be
a probability (i.e., its value falls in the interval \([0,1]\)):
\begin{proposition}
\label{prp:rn-prob-strict-btw-01}
Under the no-arbitrage principle, the risk-neutral probability \(\displaystyle
q=\frac{e^{(r-\delta)h}-d}{u-d}\) satisfies \(0<q<1\).
\end{proposition}
\begin{pf}
Firstly, note that
\[
0<q=\frac{e^{(r-\delta)h}-d}{u-d}<1
\iff
d<e^{(r-\delta)h}<u.
\]
Now assume to the contrary that \(d\ge e^{(r-\delta)h}\) or \(u\le
e^{(r-\delta)}h\).  We shall only give a proof for the case \(d\ge
e^{(r-\delta)h}\) as the proof for another case is analogous (just consider
reverse of the strategy below).

In this case, we have
\[
u>d\ge e^{(r-\delta)h}\implies e^{\delta h}S_u>e^{\delta h}S_d\ge S_0e^{rh}.
\]
Now consider the following strategy:
\begin{center}
\begin{tabular}{ccc}
\toprule
Time&Transaction&Cash flow\\
\midrule
0&short bond (borrow \(S_0\))&\(+S_0\)\\
&buy 1 share of stock \faIcon{apple-alt}&\(-S_0\)\\
&&Total: \(0\) \\
\bottomrule
\end{tabular}
\end{center}
The payoffs at time \(h\) are given by:
\begin{center}
\begin{tabular}{cccc}
\toprule
Case&Long \faIcon{apple-alt}&Short Bond&Total\\
\midrule
\(S_h=S_u\)&\(e^{\delta h}S_u\)&\(-S_0e^{rh}\)&\(e^{\delta h}S_u-S_0e^{rh}>0\) \\
\(S_h=S_d\)&\(e^{\delta h}S_d\)&\(-S_0e^{rh}\)&\(e^{\delta h}S_d-S_0e^{rh}\ge 0\) \\
\bottomrule
\end{tabular}
\end{center}
Hence this is an arbitrage.
\end{pf}

\begin{note}
This result also suggests a lower bound on \(u\) and an upper bound on \(d\)
(both are of value \(e^{(r-\delta)h}\)): The growth factors \(u\) and \(d\)
cannot be too low and too high respectively.

In other words, to construct a one-period model that is consistent with the
no-arbitrage principle, these bounds on \(u\) and \(d\) need to be satisfied.
 \end{note}
\item \label{it:rn-pricing-fmla}
Thus, it is reasonable to ``treat'' \(q\) as a probability. To perform
risk-neutral pricing, we work in an ``imaginary'' \emph{risk-neutral world}
where the \emph{actual} ``up'' probability is \(q\) rather than \(p\) (it is
like a ``parallel universe''). We shall use the notation \(\expvq{\cdot}\) to
denote the expectation in a risk-neutral world.

Using this notation, we can write the risk-neutral pricing formula in
\labelcref{it:price-by-rep-fmla-exp-form} as
\[
V_0=e^{-rh}\expvq{\text{terminal payoff}}.
\]
\subsubsection*{Further Explanation on Risk-Neutral Pricing}
(Reference: \textcite[Section~4.1.2]{lo2018derivative})

\item Consider the real world (the world we live in) and a risk-neutral
world:

\begin{center}
\begin{tikzpicture}
\node[] (s0) at (0,0) {\(S_0\)};
\node[] (su) at (4,1) {\(S_u=S_0u\)};
\node[] (sd) at (4,-1) {\(S_d=S_0d\)};
\draw[-Latex] (s0.east) -- (su.west)
node[midway, above=0.3]{\(p\)};
\draw[-Latex] (s0.east) -- (sd.west)
node[midway, below=0.3]{\(1-p\)};
\node[] () at (0,2) {Time 0};
\node[] () at (4,2) {Time \(h\)};
\node[] () at (6,2) {Payoff};
\node[] () at (6,1) {\(V_u\)};
\node[] () at (6,-1) {\(V_d\)};
\node[] () at (3,3) {\textbf{Real World}};
\end{tikzpicture}

\begin{tikzpicture}
\node[] (s0) at (0,0) {\(S_0\)};
\node[] (su) at (4,1) {\(S_u=S_0u\)};
\node[] (sd) at (4,-1) {\(S_d=S_0d\)};
\draw[-Latex] (s0.east) -- (su.west)
node[midway, above=0.3]{\(q\)};
\draw[-Latex] (s0.east) -- (sd.west)
node[midway, below=0.3]{\(1-q\)}
;
\node[] () at (0,2) {Time 0};
\node[] () at (4,2) {Time \(h\)};
\node[] () at (6,2) {Payoff};
\node[] () at (6,1) {\(V_u\)};
\node[] () at (6,-1) {\(V_d\)};
\node[] () at (3,3) {\textbf{Risk-Neutral World}};
\draw[fill, ForestGreen, opacity=0.2] (-2,-2) rectangle (8,4);
\end{tikzpicture}
\end{center}
\begin{note}
To contrast the probabilities in real world and risk-neutral world, we call the
probability \(p\) as \defn{true probability} (of ``up''), and the probability
\(q\) is called \emph{risk-neutral probability} (of ``up'').
\end{note}

\item Note that the construction of replicating portfolio for the derivative
\faIcon{scroll} does \emph{not} depend on the ``up'' probability (the
expressions in \labelcref{it:rep-port} are free of \(p\)). Hence, in
\emph{both} real and risk-neutral worlds, we would construct \emph{the same
replicating portfolio} for \faIcon{scroll} (i.e., the portfolio with \(\Delta\)
shares of \faIcon{apple-alt} and \(B\) in bond as suggested in
\labelcref{it:rep-port}).

\item Consequently, by the law of one price, we would have the same time-0 price
for \faIcon{scroll} in both worlds (which equals the price of the common
replicating portfolio). Therefore, to find the time-0 price of \faIcon{scroll}
in real world (what we want), it suffices to find the price in
\emph{risk-neutral world} --- a key idea in \emph{risk-neutral pricing}.

\item More precisely, a \defn{risk-neutral world} is a world where all
investors are \defn{risk-neutral}, i.e., they only care about the expected
returns on their investments, but not their \emph{risks} \faIcon{arrow-right}
the (required) expected return rate for every financial instrument is the
risk-free rate (there is not \emph{risk premium}!).

\begin{note}
It ``turns out'' that the appropriate probability of ``up'' for a risk-neutral
world can only be \(q\), hence it is called \emph{risk-neutral probability}.
\end{note}


\item \label{it:rn-pricing-fmla-deriv}
Particularly, the expected return rate of the derivative \faIcon{scroll}
in the risk-neutral world would be the risk-free rate:
\[
\expvq{\frac{V_h-V_0}{V_0}}=e^{rh}-1\implies
V_0=e^{-rh}\expvq{V_h}
=e^{-rh}\expvq{\text{terminal payoff}}.
\]
\begin{note}
The time-0 price of the derivative \(V_0\) is nonrandom (it is the time-0
price of the replicating portfolio, which is deterministic).
\end{note}

\item For the expected return rate of \faIcon{scroll} in the \emph{real world},
it is \emph{higher} than the risk-free rate due to the presence of \emph{risk
premium}. (Investors in the real world are not risk-neutral, and most are
\emph{risk-averse}. See STAT3904 for more details.)

\item Let \(\gamma\) be the annual continuously compounded expected return rate
for \faIcon{scroll} in the real world (which is higher than \(r\)). Then, the
``real-world pricing formula'' is
\[
V_0=e^{-\gamma h}\expv{\text{terminal payoff}}
\]
(derived using similar argument as \labelcref{it:rn-pricing-fmla-deriv}).

However, it is often difficult to be used in practice since it is usually hard
to get/estimate the values of the true probabilities (applicable in the real
world) and \(\gamma\). On the other hand, the values of \(q\) (which is free of
\(p\)) and \(r\) are often more feasible to be obtained/estimated. This shows
the elegance of risk-neutral pricing.

\item Note that a crucial part of this explanation is the construction of
replicating portfolio --- So, the argument here is not a \emph{replacement} of
pricing by replication. Rather, it is \emph{based on} pricing by replication.

For other models where similar pricing by replication is possible, the argument
provided here would work analogously. Thus, the idea of risk-neutral
pricing appears \emph{frequently}.

\item For a more detailed and formal treatment on risk-neutral pricing and
knowing more about its technical details, see STAT3911.
\end{enumerate}
\subsection{Constructing a Binomial Tree}
\begin{enumerate}
\item So far, we treat the growth factors \(u\) and \(d\) for the binomial tree
as given and perform numerous calculations based on them. But in practice we do
not have a ``given'' binomial tree and in order to use the binomial option
pricing model, we need to \emph{construct} one (i.e. choosing values of \(u\)
and \(d\)) in some way.

\item \label{it:tree-ud-bounds}
From \cref{prp:rn-prob-strict-btw-01}, we know that to be consistent with
the no-arbitrage principle, the values of \(u\) and \(d\) must satisfy
\[
d<e^{(r-\delta)h}<u.
\]
But apart from this (and the model assumption that \(u>d\)), there is no other
requirement on what \(u\) and \(d\) can be.

\item Since the values of \(u\) and \(d\) control how ``risky''/``volatile'' the stock
\faIcon{apple-alt} is (when \(u\) and \(d\) are further apart, the stock
\faIcon{apple-alt} can be seen as more ``volatile''), it seems natural to
choose \(u\) and \(d\) based on the inherent ``riskiness'' of the stock
\faIcon{apple-alt}.

\item \label{it:volatility}
The ``riskiness'' of the stock \faIcon{apple-alt} can be measured by its
\emph{volatility}, which indicates the ``variability'' of the stock price. The
\defn{volatility} of the stock \faIcon{apple-alt} (denoted by
\defn{\(\sigma\)}) is the \emph{annualized} standard deviation of its
(\(h\)-year) continuously compounded \emph{price} return\footnote{That is, we
consider only the changes in \emph{price} of the stock and ignore the
continuous dividend (growth in no.\ of shares owned) in the calculation.} \(\ln
(S_h/S_0)\):

\[
\sigma=\sqrt{\frac{1}{h}\vari{\ln\frac{S_h}{S_0}}}.
\]
\begin{remark}
\item The variance \(\vari{\ln (S_h/S_0)}\) is for a period with duration \(h\)
years. Dividing it by \(h\) gives the \emph{annualized} variance (``nominal''
value for the variance for a 1-year period). Taking square root then gives the
\emph{annualized} standard deviation.

\item More generally, the \defn{\(t\)-year volatility} (denoted by
\defn{\(\sigma_t\)}) is \(\sigma_t=\sigma\sqrt{t}\). (Multiplying \(\sqrt{t}\)
converts the ``nominal'' for 1 year to ``nominal'' for \(t\) years.)
Particularly, the \(h\)-year volatility is
\[
\sigma_h=\sigma\sqrt{h}=\sqrt{\vari{\ln\frac{S_h}{S_0}}}.
\]
\item The volatility is positive since the stock \faIcon{apple-alt} is risky
\(\implies\) \(S_h\) is random \(\implies\) the variance is
positive.
\end{remark}

\item We can use the historical stock prices to \emph{estimate} the stock's
volatility \faIcon{arrow-right} the (estimated) value can then be used for
constructing a binomial tree.  A common approach for such volatility-based
construction of a binomial tree is to use a \defn{forward tree}, which is the
binomial tree with the growth factors set to be
\[
u=e^{(r-\delta)h+\sigma\sqrt{h}}\quad\text{and}\quad d=e^{(r-\delta)h-\sigma\sqrt{h}},
\]
where \(\sigma\), \(r\), and \(\delta\) are supposed to be known values
(obtained possibly by estimation based on past data or assumption).

\begin{note}
These choices of growth factors satisfy the bounds mentioned in
\labelcref{it:tree-ud-bounds}, since when \(\sigma>0\), we have
\(\sigma\sqrt{h}>0\), which implies \(e^{\sigma\sqrt{h}}>1\) and
\(e^{-\sigma\sqrt{h}}<1\) .
\end{note}
\item To see why a forward tree is called so, first recall from
\labelcref{it:perfect-mkt-fwd-price-cts-div} that the forward price for a
forward on the stock \faIcon{apple-alt} (with dividend yield \(\delta\)) with
delivery date \(h\) is
\[
F_0=S_0e^{(r-\delta)h}.
\]
Under the forward tree, the ``up'' and ``down'' time-\(h\) stock prices are:
\begin{itemize}
\item \(S_u=S_0e^{(r-\delta)h+\sigma\sqrt{h}}=F_0e^{\sigma\sqrt{h}}=F_0e^{\sigma_h}\);
\item \(S_d=S_0e^{(r-\delta)h-\sigma\sqrt{h}}=F_0e^{-\sigma\sqrt{h}}=F_0e^{-\sigma_h}\).
\end{itemize}
So, the \emph{forward} price \(F_0\) serves as a ``baseline'' for time-\(h\)
stock price to vary around (based on the volatility \(\sigma\)), and we choose
\(e^{\sigma_h}\) and \(e^{-\sigma_h}\) as the multiplicative factors for ``up''
and ``down'' respectively.
\begin{intuition}
Given a log return \(r_{\text{log}}\), the ``growth factor'' (in the context of
rate of return) is \(e^{r_{\text{log}}}\), so the prices \(S_u\) and \(S_d\)
may be \emph{informally} regarded as having ``1 s.d.\ from baseline''.
\end{intuition}

\item \label{it:fwd-tree-rn-prob-fmla}
For a forward tree, the risk-neutral probability (of ``up'') is
\[
q=\frac{e^{(r-\delta)h}-d}{u-d}
=\frac{e^{(r-\delta)h}-e^{(r-\delta)h-\sigma\sqrt{h}}}{e^{(r-\delta)h+\sigma\sqrt{h}}-e^{(r-\delta)h-\sigma\sqrt{h}}}
=\frac{1}{1+e^{\sigma\sqrt{h}}}.
\]
\begin{warning}
If you are familiar with machine learning, you may notice that the final
expression ``seems to be'' the sigmoid function with input \(\sigma\sqrt{h}\).
But it is \underline{not}. The final expression here does not have minus sign!
\end{warning}

This formula provides a convenient way to compute the risk-neutral probability
\(q\) \emph{without} knowing \(u, d, r,\) and \(\delta\).

\item However, a disadvantage of the forward tree is that we must have
\(q<\frac{1}{2}\) (since \(\sigma>0\implies e^{\sigma\sqrt{h}}>1\)), which can
be as a \emph{built-in bias} for a forward tree.
\item \label{it:estimate-vol}
Here we give an approach to estimate the volatility \(\sigma\) based on past
stock price data (which is also useful for the \emph{Black-Scholes model};
see \labelcref{it:bs-model-def}).

Suppose we observe \(n+1\) stock prices \(S_0, S_h, S_{2h},\dotsc,S_{nh}\)
(\(h\) is the time length between adjacent observations). (E.g., when
\(h=1/12\), then they are observed at consecutive \emph{monthly} intervals.)
The observed (non-annualized) continuously compounded returns over the time
intervals \([0,h),[h,2h),\dotsc,[(n-1)h,nh)\), denoted by
\(r_{1},r_{2},\dotsc,r_{n}\) (resp.), are respectively
\[
r_1=\ln \frac{S_{h}}{S_{0}}, r_2=\ln \frac{S_{2h}}{S_{h}},\dotsc, r_n=\ln
\frac{S_{nh}}{S_{(n-1)h}}.\footnote{These observations can be treated as
observations from \(n\) ``experiments'' conducted ``naturally'' in the market
at the past. At each initial time point \(i\) (the beginning of the time
interval \([i,i+h)\)), the stock price \(S_i\) is known to the market
participants, while the stock price \(S_{i+h}\) at time \(i+h\) is unknown
(random) to them at that moment \faIcon{arrow-right} the \(h\)-year
log return here ``matches'' with the one mentioned in
\labelcref{it:volatility}.

Then, each observed value \(r_i\) can be seen as an ``observation'' from that
random \(h\)-year log return in \labelcref{it:volatility} in the
``experiment''.
}
\]

\item Here we assume that the \emph{returns} are independent and identically
distributed (i.i.d.).\footnote{On the other hand, it is \emph{unreasonable} to
assume that the stock prices are i.i.d. --- they are clearly dependent!} Having
\(n\) i.i.d.\ observations (of returns) \(r_1,\dotsc,r_n\), we can use standard
techniques in statistics to perform the estimation.

\item Here we use a simple technique: \emph{method of moments}. We equate the
\emph{sample variance} of the observations \(r_1,\dotsc,r_n\) and the
(theoretical) variance of \(h\)-year continuously compounded return (where the
initial value is nonrandom): \(\sigma_h^2=\sigma^2h\) to get
\[
\frac{1}{n-1}\sum_{i=1}^{n}(r_i-\overline{r})^2=\sigma^2h,
\]
where \(\overline{r}=(1/n)\sum_{i=1}^{n}r_i\) is the sample mean of
\(r_1,\dotsc,r_n\).

\item \label{it:vol-mom-est-fmla}
Solving this equation in \(\sigma\) gives the estimate of volatility:
\[\widehat{\sigma}
=\frac{1}{\sqrt{h}}\sqrt{\frac{1}{n-1}\sum_{i=1}^{n}(r_i-\overline{r})^2},\]
which is known as \defn{historical volatility}.
\end{enumerate}
