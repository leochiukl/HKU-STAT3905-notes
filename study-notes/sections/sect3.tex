\section{Forward/Futures Price}
\label{sect:fwd-futures-price}
\begin{enumerate}
\item For convenience in wordings, we shall focus on \emph{forward} here. But all
results discussed here are also applicable to futures.
\end{enumerate}
\subsection{No-Arbitrage Principle and Perfect Market}
\begin{enumerate}
\item Recall the \emph{no-arbitrage principle} mentioned in
\labelcref{it:no-arbitrage-principle}.
\item An important consequence of the no-arbitrage principle is the \emph{law
of one price}. But before stating that, we need to discuss another critical
assumption in financial economics: \emph{perfect market}.
\item \label{it:perfect-mkt}
In a \defn{perfect market},
\begin{itemize}
\item there is no transaction cost;
\item borrowing rate and lending rate are the same;
\item credit risk does not exist;
\item short selling is always possible.
\end{itemize}
Due to these ``nice'' properties of perfect market, working with it would be
convenient, and indeed many ``nice'' results assume the market is perfect (in
addition to the no-arbitrage principle).

Unless otherwise specified, we assume that we are in a perfect market.

\begin{warning}
However, the actual market in the real world is clearly \emph{not} perfect. So
one should be careful about the potential impacts of this when applying the
results in the real world.
\end{warning}
\item \label{it:loop}
The law of one price (LOOP) is as follows:
\begin{theorem}[Law of one price]
\label{thm:loop}
Under the no-arbitrage principle and perfect market, two positions
\faIcon{scroll}\faIcon{scroll} with the same payoff at any future time point
must ``sell'' at the same price now.
\end{theorem}
\begin{remark}
\item ``Price'' and ``value'' carry the same meaning.
\begin{warning}
``Price of a forward'' (\(=0\) at time 0) and ``forward price'' (negotiated)
are \underline{not} the same! ``Price'' here is in the former sense.
\end{warning}
\item A position ``selling'' at a price \faIcon{dollar-sign} now means that it
costs \faIcon{dollar-sign} to take that position now.
\item This means there is always exactly \emph{one price} for any instrument
with a given future payoff pattern.
\item For proofs involving no-arbitrage principle like this one, a general
proof strategy is to use \emph{proof by contradiction}: First assume the result
is false, and then try to obtain an arbitrage strategy (which contradicts the
no-arbitrage principle).
\end{remark}

\begin{pf}
Assume to the contrary that the positions ``sell'' at different prices now. Let
{\color{ForestGreen}\faIcon{scroll}} and {\color{red}\faIcon{scroll}} be the
``cheaper'' and ``more expensive'' position, that ``sell'' at prices \(P\) and \(Q\)
respectively (\(P<Q\)). Then we have the following arbitrage strategy:
\begin{center}
\begin{tabular}{clr}
\toprule
Time&Transaction&Cash flow\\
\midrule
0&take the position {\color{ForestGreen}\faIcon{scroll}}
&\(-P\)\\
&\emph{reverse} ``take the {\color{red}\faIcon{scroll}} position''
&\(+Q\)\\
&&Total: \(Q-P>0\)\\
\midrule
any \(t>0\)&close out all positions &\(0\) \\
\bottomrule
\end{tabular}
\end{center}
\begin{note}
There is no cash outflow and is a risk-free profit, hence it is an arbitrage
strategy.
\end{note}

There is zero cash flow when all positions are closed out (and for the omitted
time point) since {\color{ForestGreen}\faIcon{scroll}} and the ``reverse
{\color{red}\faIcon{scroll}}'' always have offsetting payoffs, at any future
time point.
\end{pf}

\begin{remark}
\item For a given strategy (comprising of some transaction(s)), the \defn{reverse}
strategy is the one consisting of the ``countering'' transaction(s), i.e., those
created by interchanging ``buy'' \faIcon{arrows-alt-h} ``sell'', ``long''
\faIcon{arrows-alt-h} ``short'', etc. in the original transaction(s).

\item The ``no transaction cost'' assumption in the perfect market is useful for
ensuring the cash flows at time 0 are indeed \(-P\) and \(+Q\).

\item The ``short selling is always possible'' assumption is useful for
ensuring that the transactions above are indeed doable.

\item The rest of them of them are useful for ensuring that the cash flow when
closing out all positions at time \(t\) is indeed 0.
\end{remark}
\item Here is a corollary of LOOP:
\begin{corollary}
\label{cor:zero-payoff-zero-price}
Under the no-arbitrage principle and perfect market, a position \faIcon{scroll}
with zero payoff at any future time point must ``sell'' at zero price now.
\end{corollary}
\begin{pf}
Consider this position \faIcon{scroll} and another ``position'' which is about
owning ``nothing''/``blank paper''. Clearly the latter has zero payoff at any
future time also, and must ``sell'' at zero price now (under no-arbitrage
principle).  So by LOOP, the result follows.
\end{pf}

\begin{note}
If the (total) cash flow at any time point \(t>0\) involving transaction(s) is
0, the payoff/value at any future time point would be zero (under no-arbitrage
principle).
\end{note}
\end{enumerate}
\subsection{No-Arbitrage Forward/Delivery Price}
\begin{enumerate}
\item Notations:
\begin{itemize}
\item \defn{\(r\)}: the annual risk-free interest rate compounded
continuously;
\item \defn{\(F_t\)}: forward/delivery price for a forward (with delivery date
being time \(T\) always) negotiated at time \(t\).
\end{itemize}
\begin{note}
\defn{Risk-free rate} means the rate of return earned for a investment without risk,
i.e., one that \emph{guarantees} a certain future payoff pattern.
\end{note}
\item \label{it:perfect-mkt-fwd-price}
In a perfect market, there is exactly one possible delivery/forward price (for a
forward on 1 \faIcon{apple-alt}), given by:
\[
F_0=S_0e^{rT}.
\]
\begin{remark}
\item Although this result is for forwards on ``one unit'' of underlying asset
\faIcon{apple-alt}, one may modify the unit to apply this result for forwards
on any number of underlying asset. Example: ``one unit'' of \faIcon{apple-alt}
= 1000 apples \faIcon{arrow-right} the result is applicable for a forward on
1000 apples (\(S_0\) is spot price of 1000 apples).
\item Furthermore, by changing the time labelling (i.e., modifying the
definition of ``now''), this result can still be applied for a forward
negotiated at time \(t\):
\[
F_t=S_te^{r(T-t)}.
\]
In words,
\[
\text{negotiated forward price} = \text{``current'' (wrt time \(t\)) spot price} \times
e^{r(\text{time length until maturity})}.
\]
\item The above remarks apply similarly to later results.
\end{remark}

\begin{pf}
We make use of the \cref{cor:zero-payoff-zero-price}. Consider the following
strategy:
\begin{center}
\begin{tabular}{clr}
\toprule
Time&Transaction&Cash flow\\
\midrule
0&borrow \(F_0e^{-rT}\)
&\(+F_0e^{-rT}\)\\
&buy 1 \faIcon{apple-alt} at spot price
&\(-S_0\)\\
&short the forward
&\(0\)\\
&&Total: \(F_0e^{-rT}-S_0\)\\
\midrule
\(T\)&sell 1 \faIcon{apple-alt} at delivery price& \(+F_0\) \\
&repay the loan& \(-F_0\)\\
&&Total: \(0\)\\
\bottomrule
\end{tabular}
\end{center}
\begin{note}
Any loan is risk-free here since we assume that credit risk does not exist in a
perfect market.
\end{note}

Hence we must have \(F_0e^{-rT}-S_0=0\), as desired.
\end{pf}

\begin{note}
The general idea in constructing this kind of proof is trying to design the
transactions such that the (total) payoff at any future time point is 0. Then,
\cref{cor:zero-payoff-zero-price} can be readily used.
\end{note}


\item The consequence of \labelcref{it:perfect-mkt-fwd-price} is that, in a
perfect market, if the forward price \(F_0\) is \emph{not}
\(S_0e^{rT}\), there would be an arbitrage opportunity.

\item Naturally, we would be interested in \emph{how} to capture such arbitrage
opportunity in such case. It turns out that the strategies of capturing such
opportunities are so ``famous'' that they have names: ``cash-and-carry'' and
``reverse cash-and-carry''.

\item \defn{Cash-and-carry} (C\&C) strategy (for the case where
\(F_0>S_0e^{rT}\): forward is ``overpriced''):
\begin{center}
\begin{tabular}{clr}
\toprule
Time&Transaction&Cash flow\\
\midrule
0&borrow \(F_0e^{-rT}\)
&\(+F_0e^{-rT}\)\\
&buy 1 \faIcon{apple-alt} at spot price
&\(-S_0\)\\
&short the forward
&\(0\)\\
&&Total: \(F_0e^{-rT}-S_0>0\)\\
\midrule
\(T\)&sell 1 \faIcon{apple-alt} at delivery price& \(+F_0\) \\
&repay the loan& \(-F_0\)\\
&&Total: \(0\)\\
\bottomrule
\end{tabular}
\end{center}
\begin{note}
We call this ``cash-and-carry'' since at time 0 we borrow ``cash'', and then
we ``carry'' 1 \faIcon{apple-alt} from time 0 to \(T\).
\end{note}
\item \defn{Reverse cash-and-carry} (RC\&C) strategy (for the case where
\(F_0<S_0e^{rT}\): forward is ``underpriced''):
\begin{center}
\begin{tabular}{clr}
\toprule
Time&Transaction&Cash flow\\
\midrule
0&lend \(F_0e^{-rT}\)
&\(-F_0e^{-rT}\)\\
&(short) sell 1 \faIcon{apple-alt} at spot price
&\(+S_0\)\\
&long the forward
&\(0\)\\
&&Total: \(S_0-F_0e^{-rT}>0\)\\
\midrule
\(T\)&collect proceeds from the loan& \(+F_0\)\\
&buy 1 \faIcon{apple-alt} at delivery price& \(-F_0\) \\
&repay the short sale (i.e., return 1 \faIcon{apple-alt} to the lender \faIcon{landmark})& \(0\) \\
&&Total: \(0\)\\
\bottomrule
\end{tabular}
\end{center}
\begin{note}
As its name suggests, this strategy is the reverse strategy for cash-and-carry.
Indeed, often when we figure out an strategy to capture the arbitrage
opportunity from mispricing at a specific direction, we can use its
\emph{reverse strategy} to capture the arbitrage opportunity from mispricing at
\emph{another} direction.
\end{note}
\end{enumerate}
\subsection{Imperfect Market: Borrowing Rate \(>\) Lending Rate}
\begin{enumerate}
\item In practice, borrowing rate often exceeds lending rate (resulting in an
imperfect market).
\item Notations:
\begin{itemize}
\item \(r_B\): borrowing rate (annual, compounded continuously);
\item \(r_L\): lending rate (annual, compounded continuously).
\end{itemize}
\item \label{it:imper-fwd-price-interval}
When \(r_B>r_L\) (while other assumptions for a perfect market are satisfied),
there are more than one possible forward price \(F_0\) (here LOOP is not
applicable as the market is imperfect). It can be any value lying in the price
interval:
\[
\qty[S_0e^{r_LT},S_0e^{r_BT}].
\]
\begin{pf}
When \(F_0>S_0e^{r_BT}\) (\(F_0<S_0e^{r_LT}\)), the C\&C (RC\&C resp.) strategy
is an arbitrage strategy.
\begin{note}
\(r_B\) (\(r_L\)) is the rate applicable when we borrow (lend) money. So in the
C\&C (RC\&C) strategy, the first transaction becomes ``borrow \(F_0e^{-r_BT}\)''
(``lend \(F_0e^{-r_LT}\)'').
\end{note}
\end{pf}
\end{enumerate}
\subsection{Forward on a Stock With Discrete Dividends}
\begin{enumerate}
\item Now consider \faIcon{apple-alt} as a stock with discrete dividends. The
current stock (spot) price is \(S_0\), and it is expected that
\faIcon{apple-alt} will make dividend payment \faIcon{money-bill-wave}
\(D_{t_i}\) at time \(t_i\), \(i=1,\dotsc,n\), where \(0<t_1<\dotsb<t_n<T\).

\item \label{it:perfect-mkt-fwd-price-disc-div}
To ``incorporate'' the effects of dividends, the (unique) price of
forward on \faIcon{apple-alt} in a \emph{perfect market} needs to be adjusted
to:
\[
F_0=S_0e^{rT}-\sum_{i=1}^{n}D_{t_i}e^{r(T-t_i)}.
\]
\begin{note}
In words, it is
\[
\text{original \(F_0\)}-\text{FV of all dividend payments at time \(T\) (at risk-free rate)}.
\]
\end{note}

\begin{pf}
Consider the following strategy:
\begin{center}
\begin{tabular}{clr}
\toprule
Time&Transaction&Cash flow\\
\midrule
0&borrow \(e^{-rT}\qty(F_0+\sum_{i=1}^{n}D_{t_i}e^{r(T-t_i)})\)
&\(+e^{-rT}\qty(F_0+\sum_{i=1}^{n}D_{t_i}e^{r(T-t_i)})\)\\
&buy 1 \faIcon{apple-alt} at spot price
&\(-S_0\)\\
&short the forward
&\(0\)\\
&&Total: \(e^{-rT}\qty(F_0+\sum_{i=1}^{n}D_{t_i}e^{r(T-t_i)})-S_0\)\\
\midrule
\(t_1\)&receive dividend payment \(D_{t_1}\)&\(+D_{t_1}\)\\
&lend \(D_{t_1}\)&\(-D_{t_1}\)\\
&&Total: \(0\)\\
\midrule
\vdots&\vdots&\vdots\\
\midrule
\(t_n\)&receive dividend payment \(D_{t_n}\)&\(+D_{t_n}\)\\
&lend \(D_{t_n}\)&\(-D_{t_n}\)\\
&&Total: \(0\)\\
\midrule
\(T\)&sell 1 \faIcon{apple-alt} at delivery price& \(+F_0\) \\
&collect proceeds from the loans (the ones at time \(t_1,\dotsc,t_n\))& \(+\sum_{i=1}^{n}D_{t_i}e^{r(T-t_i)}\)\\
&repay the loan (the one at time 0)& \(-F_0-\sum_{i=1}^{n}D_{t_i}e^{r(T-t_i)}\)\\
&&Total: \(0\)\\
\bottomrule
\end{tabular}
\end{center}
Hence we must have
\(e^{-rT}\qty(F_0+\sum_{i=1}^{n}D_{t_i}e^{r(T-t_i)})-S_0=0\), as desired.
\end{pf}
\end{enumerate}

\subsection{Forward on a Stock With Continuous Dividends}
\begin{enumerate}
\item For mathematical convenience, sometimes we choose to model the dividends
as being paid \emph{continuously} (at a rate proportional to the stock price)
rather than in a discrete manner. Such rate is called (annual) \defn{dividend
yield}, denoted by \defn{\(\delta\)}.
\item The meaning of dividend yield is illustrated below:
\begin{center}
\begin{tikzpicture}
\begin{axis}[domain=0:80, ylabel={No.\ of \faIcon{apple-alt} shares (\(N_t\))}, xlabel={Time \(t\)}]
\addplot[blue]{10*e^(0.02*x)};
\draw[-Latex, violet] (20,9) -- (1,9.5)
node[pos=-0.4] {\(N_0\) \faIcon{apple-alt}};
\draw[-Latex, brown] (20,35) -- (5,12);
\draw[-Latex, brown] (20,35) -- (20,16);
\draw[-Latex, brown] (20,35) -- (35,22);
\draw[-Latex, brown] (20,35) -- (50,30);
\draw[-Latex, brown] (20,35) -- (65,40);
\node[brown] (div) at (19,38) {\(+N_tS_t\delta\,dt\)};
\node[violet] (apple) at (19,47) {\(+N_t\delta\,dt\) \faIcon{apple-alt}};
\draw[<->, thick] (div.north) -- (apple.south);
\fill[violet] (0,10) circle [radius=0.5];
\node[blue] () at (60,20) {\(N_t=N_0e^{\delta t}\)};
\end{axis}
\end{tikzpicture}
\end{center}
More explanation:
\begin{itemize}
\item \(N_t\): no. of \faIcon{apple-alt} shares we own at time \(t\)
\item For every ``infinitesimal'' time interval \([t,t+dt]\), we receive
dividend payment \faIcon{money-bill-wave} \(S_{t}\delta\,dt\) per shares
\faIcon{arrow-right} total amount we receive is \(N_tS_t\delta\,dt\).
\item Reinvesting this amount in the stock \faIcon{apple-alt} adds
\(\displaystyle \frac{N_tS_t\delta\,dt}{S_t}=N_t\delta\,dt\) \faIcon{apple-alt}
shares to the shares we own \faIcon{arrow-right}
\[dN_t=N_t\delta\,dt \implies
\frac{dN_t}{dt}=N_t\delta.\]
\item Solving this ODE gives \(N_t=N_0e^{\delta t}\).
\end{itemize}
\begin{note}
We shall assume automatic dividend reinvestment in this notes. So we would own
more and more \faIcon{apple-alt} shares as time passes, when \faIcon{apple-alt}
has continuous dividends.
\end{note}
\item \label{it:perfect-mkt-fwd-price-cts-div}
In a perfect market, there is exactly one possible delivery price (for a forward
on 1 \faIcon{apple-alt}), given by:
\[
F_0=S_0e^{(r-\delta)T}.
\]
\begin{pf}
Consider the following strategy:
\begin{center}
\begin{tabular}{clr}
\toprule
Time&Transaction&Cash flow\\
\midrule
0&borrow \(F_0e^{-rT}\)&\(+F_0e^{-rT}\)\\
&buy \(e^{-\delta T}\) \faIcon{apple-alt} at spot price&\(-S_0e^{-\delta T}\)\\
&short the forward&\(0\)\\
&&Total: \(F_0e^{-rT}-S_0e^{-\delta T}\)\\
\midrule
\(t\in(0,T)\)&\makecell{receive dividend payments continuously\\
(which are being reinvested in \faIcon{apple-alt})}&\(0\)\\
\midrule
\(T\)&sell 1 \faIcon{apple-alt} at delivery price& \(+F_0\) \\
&repay the loan& \(-F_0\)\\
&&Total: \(0\)\\
\bottomrule
\end{tabular}
\end{center}
\begin{note}
The number of \faIcon{apple-alt} we own accumulates from \(e^{-\delta T}\) at
time 0 to 1 at time \(T\).
\end{note}
This means
\[
F_0e^{-rT}-S_0e^{-\delta T}=0,
\]
as desired.
\end{pf}
\end{enumerate}
\subsection{Forward Price in the Presence of Storage Cost}
\begin{enumerate}
\item For some assets (e.g., diamond \faIcon[regular]{gem}), there are storage costs.

\item When an investor \faIcon{user} short sells an asset \faIcon[regular]{gem} with
storage costs (then \faIcon{user} needs to borrow \faIcon[regular]{gem} from a third
party \faIcon{landmark}), \faIcon{landmark} needs to pay to \faIcon{user} the
storage costs that would normally be incurred on the shorted \faIcon[regular]{gem}
(since the owner of \faIcon[regular]{gem} (\faIcon{landmark}), not the short-seller,
should bear the ``responsibility'' of paying storage costs).

\item Consider a forward on 1 \faIcon[regular]{gem}. Let \defn{\(C\)} be \emph{present
value} of all storage costs associated with \faIcon[regular]{gem} (at risk-free
rate). Then, by ``paying'' \(C\) now, we can have just enough money to pay all
the storage costs (some possibly in the future), by lending \(C\) and
collecting ``parts'' of proceeds from the loan at suitable time points.

\item \label{it:perfect-mkt-fwd-price-storage}
In a perfect market, there is exactly one possible delivery price (for a forward
on 1 \faIcon[regular]{gem}), given by:
\[
F_0=(S_0+C)e^{rT}.
\]
\begin{pf}
Consider the following strategy:
\begin{center}
\begin{tabular}{clr}
\toprule
Time&Transaction&Cash flow\\
\midrule
0&borrow \(F_0e^{-rT}\)&\(+F_0e^{-rT}\)\\
&buy 1 \faIcon[regular]{gem} at spot price&\(-S_0\)\\
&``pay'' the storage costs&\(-C\)\\
&short the forward&\(0\)\\
&&Total: \(F_0e^{-rT}-(S_0+C)\)\\
\midrule
\(T\)&sell 1 \faIcon[regular]{gem} at delivery price& \(+F_0\) \\
&repay the loan& \(-F_0\)\\
&&Total: \(0\)\\
\bottomrule
\end{tabular}
\end{center}
Hence, we have \(F_0e^{-rT}-(S_0+C)=0\), as desired.
\end{pf}
\end{enumerate}
\subsection{Currency Forward}
\begin{enumerate}
\item Sometimes the underlying asset of a forward is a certain number of
currency \faIcon{euro-sign}. Such forward is known as a \defn{currency forward}.
\item Before proceeding further, let us first briefly introduce foreign
exchange (forex/FX). The main idea of FX is illustrated below:
\begin{center}
\begin{tikzpicture}
\node[font=\large] () at (2.5,3) {{\color{violet}USD}/{\color{brown}JPY}: 160 (FX quote convention)};
\node[font=\large, draw, violet] (usd) at (0,0) {1 USD ({\color{violet}\faIcon{dollar-sign}})};
\node[font=\large, draw, brown] (jpy) at (5,0) {160 JPY ({\color{brown}\faIcon{yen-sign}})};
\draw[-Latex] (usd.north) to[bend left] (jpy.north);
\draw[-Latex] (jpy.south) to[bend left] (usd.south);
\node[] () at (2.5,1.5) {buy 160 {\color{brown}\faIcon{yen-sign}} using 1 {\color{violet}\faIcon{dollar-sign}} /
sell 1 {\color{violet}\faIcon{dollar-sign}} for 160 {\color{brown}\faIcon{yen-sign}}};
\node[] () at (2.5,2) {(convert {\color{violet}\faIcon{dollar-sign}} to {\color{brown}\faIcon{yen-sign}})};
\node[] () at (2.5,-1.5) {buy 1 {\color{violet}\faIcon{dollar-sign}} using 160 {\color{brown}\faIcon{yen-sign}}
/ sell 160 {\color{brown}\faIcon{yen-sign}} for 1 {\color{violet}\faIcon{dollar-sign}}};
\node[] () at (2.5,-2) {(convert {\color{brown}\faIcon{yen-sign}} to {\color{violet}\faIcon{dollar-sign}})};
\end{tikzpicture}
\end{center}
\begin{warning}
The expression ``USD/JPY'' may be a bit misleading, since the number it is
referring to is indeed how many \emph{JPY} can be converted \emph{per}
\emph{USD}, but ``/'' is often regarded as ``per'' (which is \underline{not}
the case here!).

To avoid this, one may consider using some alternative (less popular though)
quotations like ``USDJPY'' or ``USD-JPY''.
\end{warning}

Terminologies (based on the setting here):
\begin{itemize}
\item {\color{violet}USD (\faIcon{dollar-sign})} is the \defn{foreign currency}.
\item {\color{brown}JPY (\faIcon{yen-sign})} is the \defn{domestic currency}.

\begin{warning}
The definition of domestic/foreign currency is \underline{not} related to where
you live! Instead, it only depends on the format of the FX quotation.
\end{warning}
\item \defn{Exchange rate} is the price of one unit of foreign currency in
terms of (or \emph{denominated in}) domestic currency. (Here it is 160: It
costs 160 {\color{brown}\faIcon{yen-sign}} to buy 1
{\color{violet}\faIcon{dollar-sign}}.)
\end{itemize}

\item For a currency forward on a number of \faIcon{euro-sign}, its
forward/delivery price is expressed as an exchange rate (which is called
\defn{forward exchange rate}): price of 1 \faIcon{euro-sign} (foreign currency)
denominated in the domestic currency.

\item Notations:
\begin{itemize}
\item \defn{\(r_d\)} (\defn{\(r_f\)}): annual risk-free interest rate of the domestic
(foreign) currency, compounded continuously
\item (for emphasis) \(S_t\): time-\(t\) spot exchange rate (time-\(t\) price of one unit of
the foreign currency denominated in the domestic currency)
\end{itemize}

\item We fix our consideration on a kind of exchange rate (through which the
foreign and domestic currencies are fixed), say FOR/DOM (FOR: foreign currency;
DOM: domestic currency).

\item \label{it:perfect-mkt-fwd-ex-rate}
Consider a currency forward on 1 FOR. In a perfect market, there is exactly one
possible forward exchange rate (FOR/DOM), given by:
\[
F_0=S_0e^{(r_d-r_f)T}.
\]
\begin{note}
This result is applicable to currency forwards on any number of FOR after some
modification. For a currency forward on \(k\) FOR, the only possible forward
exchange rate (FOR/DOM) is \(kS_0e^{(r_d-r_f)T}\) (using similar argument as
the proof below).
\end{note}

\begin{pf}
Consider the following strategy:

\begin{center}
\begin{tabular}{clr}
\toprule
Time&Transaction&Cash flow\\
\midrule
0&borrow \(F_0e^{-r_dT}\) DOM &\(+F_0e^{-r_dT}\) DOM\\
&buy \(e^{-r_fT}\) FOR at spot exchange rate (FOR/DOM) &\(-S_0e^{-r_fT}\text{ DOM} + e^{-r_fT}\text{ FOR}\)\\
&lend \(e^{-r_fT}\) FOR&\(-e^{-r_fT}\) FOR\\
&short the forward&\(0\)\\
&&Total: \(F_0e^{-r_dT}-S_0e^{-r_fT}\) DOM\\
\midrule
\(T\)&collect proceeds from loan (lending rate: \(r_f\))& \(+1\) FOR \\
&sell 1 FOR at forward exchange rate (FOR/DOM) & \(-1\text{ FOR}+F_0\text{ DOM}\) \\
&repay the loan (borrowing rate: \(r_d\))& \(-F_0\) DOM \\
&&Total: \(0\)\\
\bottomrule
\end{tabular}
\end{center}
Thus, \(F_0e^{-r_dT}-S_0e^{-r_fT}=0\), as desired.
\end{pf}

\item Another perspective (as a special case of
\labelcref{it:perfect-mkt-fwd-price-cts-div}): an unit of ``FOR'' may be viewed
as a share of ``stock'' with continuous dividends, and the forward is
denominated in DOM:
\begin{itemize}
\item We treat the continuous \emph{interest} payments (from lending FOR) as
``dividend'' payments.
\item The dividend yield is \(r_f\) since the number of ``FOR'' we own at time
\(t\) (\(N_t\)) is \(N_0 e^{r_ft}\) (the ``dividends'' are already in terms of
FOR, hence ``reinvested'' automatically).
\item Since the forward is denominated in DOM, the risk-free rate \(r\) in
\labelcref{it:perfect-mkt-fwd-price-cts-div} is \(r_d\).
\end{itemize}
As a result, for a forward on 1 FOR (``1 \faIcon{apple-alt}''), the only
possible ``forward price'' (forward exchange rate here) is
\[
F_0=S_0e^{(r_d-r_f)T}.
\]
\begin{note}
An advantage of the proof approach in \labelcref{it:perfect-mkt-fwd-ex-rate} is
that it also hints how to capture arbitrage opportunity once it arises.
\end{note}
\end{enumerate}
\subsection{Value of a Forward}
\label{subsect:value-fwd}
\begin{enumerate}
\item Recall from \labelcref{it:fwd-cost-zero} that the time-0 value of any
forward negotiated at time 0 is zero. But, the value of such forward at time
\(t>0\) may no longer be zero.

\subsubsection*{Motivating Example}
\item We are in a perfect market and now is time 1.
Suppose we longed a forward on 1 \faIcon{apple-alt} (with no dividends, no
storage costs, etc.) at time 0, where \(S_0=10\), \(r=0.05\), and \(T=10\).
Then, we know the forward price negotiated was \(F_0=10e^{0.05(10)}\approx
\fpeval{round(10*exp(0.05*10),4)}\).  ``Holding'' that forward contract
\faIcon{scroll}, we are able (obligated indeed) to buy 1 \faIcon{apple-alt} at the
price of \fpeval{round(10*exp(0.05*10),4)}, at time \(T=10\).

Now, at time 1, the spot price of \faIcon{apple-alt} surges \faIcon{chart-line}
to 1000. So for a forward on 1 \faIcon{apple-alt} negotiated now, the forward
price is \(F_1=1000e^{0.05(9)}\approx \fpeval{round(1000*exp(0.05*9),4)}\),
which is much higher than \(F_0\).

Then, intuitively, the forward contract
\faIcon{scroll} we currently hold (that permits us to buy 1 \faIcon{apple-alt}
at time \(T\) with such a low price of \fpeval{round(10*exp(0.05*10),4)})
becomes very ``lucrative'', and hence its value at time 1 should be quite high
(i.e., we can ``sell'' \faIcon{scroll} at a high price).

\item Now consider a ``parallel universe'' where the spot price of
\faIcon{apple-alt} falls \faIcon{chart-area} to 0.001 at time 1
(\faIcon{apple-alt} becomes almost worthless!). The forward price for a forward
on 1 \faIcon{apple-alt} negotiated now would be \(F_1=0.001e^{0.05(9)}\approx
\fpeval{round(0.001*exp(0.05*9),6)}\), much lower than \(F_0\).

In such case, unfortunately the forward contract \faIcon{scroll} we currently
hold (that requires us to buy 1 \faIcon{apple-alt} at time \(T\) with such a
high price of \fpeval{round(10*exp(0.05*10),4)}) becomes a burden.
Intuitively, to get rid of this ``burden'', we need to provide compensation to
others.  This suggests the value of \faIcon{scroll} at time 1 should be
negative.

\subsubsection*{Formula}
\item After having some intuitive idea about the time-\(t\) value of a
forward negotiated at time 0, here we give an argument to derive the formula
for calculating the value.
\item \label{it:fwd-val-fmla}
Consider a forward \faIcon{scroll} (on 1 \faIcon{apple-alt}) negotiated at time
0, and a time point \(t<T\). Then, in a perfect market, the value of the
forward \faIcon{scroll} \footnote{That is, the value of a long position in the
forward.} at time \(t\) is
\[
(F_t-F_0)e^{-r(T-t)}.
\]
\begin{pf}
Let \(V_t\) be its value (or ``price'') at time \(t\). Then, consider the following strategy:
\begin{table}[!h]
\centering
\sffamily
\begin{tabular}{clr}
\toprule
Time&Transaction&Cash flow\\
\midrule
\(t\)&borrow \((F_t-F_0)e^{-r(T-t)}\)\tablefootnote{If \(F_t-F_0<0\), this is to be replaced by ``\emph{lend} \((F_0-F_t)e^{-r(T-t)}\)''.}&\(+(F_t-F_0)e^{-r(T-t)}\)\\
&short a forward (on 1 \faIcon{apple-alt}) negotiated at time \(t\)&\(0\)\\
&long that forward \faIcon{scroll} (negotiated at time 0)&\(-V_t\)\\
&&Total: \((F_t-F_0)e^{-r(T-t)}-V_t\)\\
\midrule
\(T\)&buy 1 \faIcon{apple-alt} at the delivery price \(F_0\)\tablefootnote{possibly on a loan that is to be repaid immediately by income(s) received from other transaction(s) at time \(T\)}& \(-F_0\) \\
&sell 1 \faIcon{apple-alt} at the delivery price \(F_t\)& \(+F_t\) \\
&repay the loan\tablefootnote{This is to be replaced by ``collect proceeds from the loan'' if \(F_t-F_0<0\).}& \(-(F_t-F_0)\) \\
&&Total: \(0\)\\
\bottomrule
\end{tabular}
\end{table}

It follows that \((F_t-F_0)e^{-r(T-t)}-V_t=0\), as desired.
\end{pf}
\end{enumerate}
